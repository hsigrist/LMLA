\documentclass[12pt,addpoints,x11names]{exam}
\usepackage{pgf,tikz}
\usepackage{multicol}
\usepackage{tkz-euclide}
\usepackage{fourier}
\usepackage{fontspec}
\usepackage{graphicx}
\usepackage{amssymb,amsmath}
\usepackage{polyglossia}
\setdefaultlanguage{spanish}
\usetikzlibrary{arrows}
\usepackage{siunitx}
\usepackage{xcolor}
\usepackage{multicol}
\usepackage{hyperref}
\usetkzobj{all}
\usepackage[left=1.5cm,right=1.5cm,top=2.5cm,bottom=1.5cm,paperheight=33cm]{geometry}
\setromanfont[Mapping={tex-text}]{Linux Biolinum O}
\setsansfont[Mapping={tex-text}]{Cantarell}
\setmonofont[Mapping={tex-text},Scale=0.8]{Pragmata Pro}
\graphicspath{{/home/hsigrist/Dropbox/images/}}
\setlength\columnsep{30pt}
\everymath{\displaystyle}
\def\NN{\mathbb{N}}
\def\RR{\mathbb{R}}
\def\QQ{\mathbb{Q}}
\def\ZZ{\mathbb{Z}}
\def\II{\mathbb{I}}
\newcommand{\framedhref}[2]{\href{#1}{\fcolorbox{SlateGray1}{SlateGray1}{#2}}}
\everymath{\displaystyle}
\extraheadheight{1in}
\extrafootheight{.75in}
\def\mytitle{Guía 2 - Geometría Analítica / Tercero Medio HC}
\def\dpto{Dpto. Matemática - LMLA $\cdot$ 2018, pág. \thepage}
\def\mykeywords{geometría cartesiana, geometría analítica}
\def\mysubject{geometría analítica}
\def\myauthor{Prof. Hans Sigrist}
\pagestyle{headandfoot}
\firstpageheader{\includegraphics[scale=.56]{logolmlabw}}
{\Large\textbf{Guía 2}\\
  Geometría Analítica\\
  Tercero Medio HC\\
  marzo 2018}
{}
\runningheader{\mytitle}{}{\dpto}
\footer{}{}{}
\definecolor{links}{HTML}{000000}
\hypersetup{%
  colorlinks,%
  linkcolor=,%
  citecolor=black,%
  urlcolor=links,%
  pdftitle={\mytitle},%
  pdfauthor={\myauthor},%
  pdfsubject={\mysubject},%
  pdfkeywords={\mykeywords}%
}
% \printanswers
\newcommand*\Matching[1]{%
  \ifprintanswers%
  \textbf{#1}%
  \else%
  \rule{2.1in}{0.5pt}%
  \fi%
}
\newlength\matchlena
\newlength\matchlenb
\settowidth\matchlena{\rule{2.1in}{0pt}}
\newcommand\MatchQuestion[2]{%
  \setlength\matchlenb{\linewidth}%
  \addtolength\matchlenb{-\matchlena}%
  \parbox[t]{\matchlena}{\Matching{#1}}\enspace\parbox[t]{\matchlenb}{#2}%
}
\begin{document}
\addpoints
\renewcommand{\solutiontitle}{\noindent\textbf{Solución:}\enspace}
\pointname{ puntos}
\pointpoints{ punto}{ puntos}
\bracketedpoints
\boxedpoints
\CorrectChoiceEmphasis{\color{red}\bfseries}
\renewcommand{\choicelabel}{\thechoice)}
\chqword{Pregunta}
\chpword{Puntos}
\chbpword{Bonus}
\chsword{Total}\hqword{Pregunta}
\hpword{Puntos}
\cellwidth{3mm}
\setlength\answerskip{2ex}
\setlength\answerlinelength{2in}
\fullwidth{\noindent\large\textbf{Nombre:}\enspace\makebox[5.2in]{\hrulefill}
  \textbf{Curso:}\enspace\makebox[0.5in]{\hrulefill}}
\fullwidth{%
  \begin{description}
  \item[Objetivos:] Obtener la distancia entre dos puntos en el plano.
    Determinar el ángulo de inclinación usando propiedades trigonómetricas.
    Determinar las ecuaciones principal y general de la recta.
  \end{description}
}
  \begin{questions}
      \fullwidth{\noindent{\large \textbf{I. Pendiente y ángulo de
            inclinación}}}
      \fullwidth{Calcula en cada caso la pendiente y el ángulo de inclinación de la
        recta que pasa por los puntos dados.}
        \question $A(4,6)$ y $B(2,3)$
        \question $A(-3,2)$ y $B(-3,5)$
        \question $A(4,8)$ y $B(-7,8)$ 
          
       \fullwidth{\noindent{\large \textbf{II. Ecuaciones principal y general de la
             recta.}}}

       \fullwidth{Encuentra la ecuación principal de la recta que pasa por los
         puntos:}
         \question $A(3,4)$ y $B(7,3)$
         \question $A(-5,2)$ y $B(-3,-1)$
         \question $A\left(\frac{1}{2},\frac{1}{3}\right)$ y $B(0,1)$
        
       \fullwidth{Encuentra la ecuación general de la recta que pasa por los
         puntos:}
         \question $A(5,-2)$ y $B(6,8)$
         \question $A(2,3)$ y $B(4,7)$
         \question $A\left(\frac{1}{4},\frac{1}{2}\right)$ y $\left(\frac{1}{8},1\right)$

       \fullwidth{Determina las ecuaciones principal y general de la recta que
         pasa por el punto dado y tiene la pendiente que se indica.}
         \question $A(6,4)$ y $m=-3$
         \question $B(0,4)$ y $m=1$
         \question $C(5,5)$ y $m=0$

       \fullwidth{Escribe las ecuaciones principal y general de la recta de modo
         que $m$ y $n$ sean, respectivamente:}
         \question $1$ y $-1$
         \question $5$ y $0$
         \question $8$ y $3$
         \question $\frac{3}{5}$ y $\frac{1}{4}$
         \question $-1$ y $2$
         \question $0$ y $2$
  
\end{questions}
\end{document}
% Local Variables:
% TeX-engine: xetex
% End:

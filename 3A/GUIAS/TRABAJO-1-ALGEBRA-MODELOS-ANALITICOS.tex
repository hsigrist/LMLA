\documentclass[12pt,addpoints,x11names]{exam}
%-------------------------------------------------------------------------------------------------------
\usepackage{pgf,tikz}
\usepackage{sectsty}
\usepackage{multicol}
\usepackage{tkz-euclide}
\usepackage{fourier}
\usepackage{fontspec}
\usepackage{graphicx}
\usepackage{amssymb,amsmath}
\usepackage{polyglossia}
\setdefaultlanguage{spanish}
\usetikzlibrary{arrows}
\usepackage{siunitx}
\usepackage{xcolor}
\usepackage{multicol}
\usepackage{hyperref}
\usetkzobj{all}
\usepackage[left=1.5cm,right=1.5cm,top=2.5cm,bottom=1.5cm,paperheight=33cm]{geometry}
%-------------------------------------------------------------------------------------------------------
\setmainfont[Mapping={tex-text}]{Linux Biolinum O}
\setsansfont[Mapping={tex-text}]{Linux Libertine O}
\setmonofont[Mapping={tex-text},Scale=0.8]{Pragmata Pro}
\allsectionsfont{\sffamily}
% -------------------------------------------------------------------------------------------------------
\graphicspath{{/home/hsigrist/Dropbox/images/}}
\setlength\columnsep{30pt}
\everymath{\displaystyle}
\def\NN{\mathbb{N}}
\def\RR{\mathbb{R}}
\def\QQ{\mathbb{Q}}
\def\ZZ{\mathbb{Z}}
\def\II{\mathbb{I}}
\newcommand{\framedhref}[2]{\href{#1}{\fcolorbox{SlateGray1}{SlateGray1}{#2}}}
\everymath{\displaystyle}
\extraheadheight{1in}
\extrafootheight{.75in}
\def\mytitle{Trabajo 1 Álgebra \& Modelos Analíticos / Tercero Medio HC}
\def\dpto{Dpto. Matemática - LMLA $\cdot$ 2018, pág. \thepage}
\def\mykeywords{geometría analítica}
\def\mysubject{geometría analítica}
\def\myauthor{Prof. Hans Sigrist}
%-------------------------------------------------------------------------------------------------------
\pagestyle{headandfoot}
\firstpageheader{\includegraphics[scale=.56]{logolmlabw}}
{\Large\textbf{Guía 2}\\
  Geometría Analítica\\
  Tercero Medio HC\\
  marzo 2018}
{}
\runningheader{\mytitle}{}{\dpto}
\footer{}{}{}
%-------------------------------------------------------------------------------------------------------
\definecolor{links}{HTML}{000000}
\hypersetup{%
  colorlinks,%
  linkcolor=,%
  citecolor=black,%
  urlcolor=links,%
  pdftitle={\mytitle},%
  pdfauthor={\myauthor},%
  pdfsubject={\mysubject},%
  pdfkeywords={\mykeywords}%
}
%-------------------------------------------------------------------------------------------------------
% \printanswers
%-------------------------------------------------------------------------------------------------------
\newcommand*\Matching[1]{%
  \ifprintanswers%
  \textbf{#1}%
  \else%
  \rule{2.1in}{0.5pt}%
  \fi%
}
\newlength\matchlena
\newlength\matchlenb
\settowidth\matchlena{\rule{2.1in}{0pt}}
\newcommand\MatchQuestion[2]{%
  \setlength\matchlenb{\linewidth}%
  \addtolength\matchlenb{-\matchlena}%
  \parbox[t]{\matchlena}{\Matching{#1}}\enspace\parbox[t]{\matchlenb}{#2}%
}
%-------------------------------------------------------------------------------------------------------
\begin{document}
\addpoints
\renewcommand{\solutiontitle}{\noindent\textbf{Solución:}\enspace}
\pointname{ puntos}
\pointpoints{ punto}{ puntos}
\bracketedpoints
\boxedpoints
\CorrectChoiceEmphasis{\color{red}\bfseries}
\renewcommand{\choicelabel}{\thechoice)}
\chqword{Pregunta}
\chpword{Puntos}
\chbpword{Bonus}
\chsword{Total}\hqword{Pregunta}
\hpword{Puntos}
\cellwidth{3mm}
\setlength\answerskip{2ex}
\setlength\answerlinelength{2in}
%-------------------------------------------------------------------------------------------------------
\fullwidth{\noindent\large\textbf{Nombre:}\enspace\makebox[5.2in]{\hrulefill}
  \textbf{Curso:}\enspace\makebox[0.5in]{\hrulefill}}
\fullwidth{%
  \begin{description}
  \item[Objetivos:] Obtener la distancia entre dos puntos en el plano.
    Determinar el ángulo de inclinación usando propiedades trigonométricas.
    Determinar las ecuaciones principal y general de la recta.
  \end{description}
}
%-------------------------------------------------------------------------------------------------------
\section{Punto medio, distancia, elementos
  del triángulo, ecuación de la recta.}

\begin{questions}
  \fullwidth{El siguiente problema requiere que observe el triángulo de la
    Figura \ref{fig:fig1}.}

  \begin{figure}[h]
    \centering
    \begin{tikzpicture}
      \tkzInit[xmax=6, ymax=4]
      \tkzAxeXY
      \tkzGrid
      \tkzClip
      \tkzDefPoint(5,1){A}
      \tkzDefPoint(2,2){B}
      \tkzDefPoint(4,3){C}
      \tkzDrawSegments(A,B B,C C,A)
      \tkzDrawPoints(A,B,C)
      \tkzLabelPoint[right](A){$A$}
      \tkzLabelPoint[left](B){$B$}
      \tkzLabelPoint[above](C){$C$}
    \end{tikzpicture}
    \caption{Triángulo $ABC$}
    \label{fig:fig1}
  \end{figure}

  \question[8] lorem jnsjxjb


















  
\end{questions}
\end{document}
% Local Variables:
% TeX-engine: xetex
% End:

\message{ !name(GUIA-1-LENGUAJE-ALGEBRAICO.tex)}\documentclass[12pt,addpoints,x11names]{exam}
\usepackage{pgf,tikz}
\usepackage{multicol}
\usepackage{tkz-euclide}
\usepackage{fourier}
\usepackage{fontspec}
\usepackage{graphicx}
\usepackage{amssymb,amsmath}
\usepackage{polyglossia}
\setdefaultlanguage{spanish}
\usetikzlibrary{arrows}
\usepackage{siunitx}
\usepackage{xcolor}
\usepackage{multicol}
\usepackage{hyperref}
\usetkzobj{all}
\usepackage[left=1.5cm,right=1.5cm,top=2.5cm,bottom=1.5cm,paperheight=33cm]{geometry}
\setromanfont[Mapping={tex-text}]{Linux Biolinum O}
\setsansfont[Mapping={tex-text}]{Cantarell}
\setmonofont[Mapping={tex-text},Scale=0.8]{Pragmata Pro}
\graphicspath{{/home/hsigrist/Dropbox/images/}}
\setlength\columnsep{30pt}
\everymath{\displaystyle}
\def\NN{\mathbb{N}}
\def\RR{\mathbb{R}}
\def\QQ{\mathbb{Q}}
\def\ZZ{\mathbb{Z}}
\def\II{\mathbb{I}}
\newcommand{\framedhref}[2]{\href{#1}{\fcolorbox{SlateGray1}{SlateGray1}{#2}}}
\everymath{\displaystyle}
\extraheadheight{1in}
\extrafootheight{.75in}
\pagestyle{headandfoot}
\firstpageheader{\includegraphics[scale=.56]{logolmlabw}}
{\Large\textbf{Guía 1}\\
  Lenguaje Algebraico\\
  Tercero Medio HC\\
  marzo 2018}
{}
\runningheader{Guía 1 - Lenguaje Algebraico / Tercero Medio HC}{}{Dpto.
  Matemática - LMLA $\cdot$ 2018, pág. \thepage}
\footer{}{}{}
\def\mytitle{Guía 1 3HC Lenguaje Algebraico}
\def\mykeywords{expresiones racionales}
\def\mysubject{lenguaje algebraico}
\def\myauthor{Prof. Hans Sigrist}
\definecolor{links}{HTML}{000000}
\hypersetup{%
  colorlinks,%
  linkcolor=,%
  citecolor=black,%
  urlcolor=links,%
  pdftitle={\mytitle},%
  pdfauthor={\myauthor},%
  pdfsubject={\mysubject},%
  pdfkeywords={\mykeywords}%
}
% \printanswers
\newcommand*\Matching[1]{%
  \ifprintanswers%
  \textbf{#1}%
  \else%
  \rule{2.1in}{0.5pt}%
  \fi%
}
\newlength\matchlena
\newlength\matchlenb
\settowidth\matchlena{\rule{2.1in}{0pt}}
\newcommand\MatchQuestion[2]{%
  \setlength\matchlenb{\linewidth}%
  \addtolength\matchlenb{-\matchlena}%
  \parbox[t]{\matchlena}{\Matching{#1}}\enspace\parbox[t]{\matchlenb}{#2}%
}
\begin{document}

\message{ !name(GUIA-1-LENGUAJE-ALGEBRAICO.tex) !offset(-3) }

\addpoints
\renewcommand{\solutiontitle}{\noindent\textbf{Solución:}\enspace}
\pointname{ puntos}
\pointpoints{ punto}{ puntos}
\bracketedpoints
\boxedpoints
\CorrectChoiceEmphasis{\color{red}\bfseries}
\renewcommand{\choicelabel}{\thechoice)}
\chqword{Pregunta}
\chpword{Puntos}
\chbpword{Bonus}
\chsword{Total}\hqword{Pregunta}
\hpword{Puntos}
\cellwidth{3mm}
\setlength\answerskip{2ex}
\setlength\answerlinelength{2in}
\fullwidth{\noindent\large\textbf{Nombre:}\enspace\makebox[5.2in]{\hrulefill}
  \textbf{Curso:}\enspace\makebox[0.5in]{\hrulefill}}
\fullwidth{%
  \begin{description}
  \item[Objetivos:] Transformar expresiones algebraicas. Operar con potencias de
    exponente fraccionario.
  \end{description}
}
\fullwidth{\noindent{\large \textbf{I. Transformación de expresiones
      algebraicas}}}
\fullwidth{Calcule el valor de las siguientes expresiones algebraicas,
  reduciendo lo mínimo lo posible.}

\begin{questions}

  \begin{multicols}{2}
    \question $\frac{2x-1}{3}+\frac{x-5}{6}+\frac{x-4}{4}=$
    
    \question $\frac{2x-3}{9}+\frac{x+2}{6}+\frac{5x+8}{12}=$
    
    \question $\frac{2x+5}{x}-\frac{x-3}{2x}-\frac{27}{8x^2}=$
    
    \question $\frac{a-b}{ab}+\frac{b-c}{bc}+\frac{c-a}{ca}=$
    
    \question $\frac{a-x}{x}+\frac{a+x}{a}-\frac{a^2-x^2}{2ax}=$

    \question $\frac{1}{x+2}+\frac{1}{x+3}=$

    \question $\frac{a}{x+a}-\frac{b}{x+b}=$

    \question $\frac{a}{x^2-4}+\frac{b}{(x-2)^2}=$

    \question $\frac{3}{x-3}+\frac{2x}{x^2-9}=$

    \question $\frac{1}{2x-3y}-\frac{x+y}{4x^2-9y^2}=$

    \question $\frac{1}{1-x^3}-\frac{1}{(1-x)^3}=$

    \question $\frac{x+a}{x-2a}-\frac{x^2+2a^2}{x^2-4a^2}=$

    \question $\frac{1}{4x-4}-\frac{1}{5x+5}+\frac{1}{1-x^2}=$

    \question $\frac{3}{1+a}-\frac{2}{1-a}-\frac{5a}{a^2-1}=$

    \question $\frac{a}{(a-b)(a-c)}+\frac{b}{(b-c)(b-a)}+\frac{c}{(c-a)(c-b)}=$

\end{multicols}

\fullwidth{\noindent{\large \textbf{II. Operar con potencias de exponente fraccionario}}}
\fullwidth{Realice las acciones que se solicitan en cada pregunta.}

\begin{multicols}{2}
  
    \question Ordene en forma creciente $\sqrt{3}$, $\sqrt[3]{6}$,
      $\sqrt[4]{10}$.
      
    \question \textbf{Exprese como radical positivo de índice $12$ la siguiente expresión:} $x^{\frac{1}{3}}=$

    \question Idem anterior $a^{-1}:a^{\frac{-1}{2}}=$

    \question Idem anterior $\sqrt[4]{ax^3}\times\sqrt[3]{a^{-1}x^{-2}}=$
      
    \question Idem anterior $\frac{1}{a^{\frac{-3}{4}}}=$

    \question Idem anterior $\frac{1}{\sqrt[8]{a^{-14}}}=$

    \question Idem anterior $\sqrt[6]{\frac{1}{a^{-2}}}=$

    \question \textbf{Exprese como radicales del mismo mínimo orden:} $\sqrt{a}$, $\sqrt[9]{a^5}$.
      
    \question Idem anterior $\sqrt[5]{a^3}$, $\sqrt{a}$.
      
    \question Idem anterior $\sqrt[8]{x^3}$, $\sqrt[9]{x^6}$, $\sqrt[20]{x^5}$.
      
    \question Idem anterior $\sqrt[16]{x^4}$, $\sqrt[12]{x^{10}}$.
      
    \question Idem anterior $\sqrt[21]{a^8b^4}$, $\sqrt[7]{ab}$.

  
\end{multicols}
\end{questions}
\end{document}

% Local Variables:
% TeX-engine: xetex
% End:

\message{ !name(GUIA-1-LENGUAJE-ALGEBRAICO.tex) !offset(-181) }

\documentclass[12pt,spanish,x11names]{beamer}
% -------------------------------------------------------------------------------------------------------
\usepackage{systeme}
\usepackage{pgfpages}
% \setbeameroption{hide notes}
% \setbeameroption{show notes}
% \setbeameroption{show notes on second screen=right}
\usetheme{Hytex}
\setbeamertemplate{navigation symbols}{}
\usecolortheme[RGB={7,29,66}]{structure}
\usepackage{tcolorbox}
\usepackage{fourier}
\usepackage{float}
\usepackage{fontspec}
\usepackage{graphicx}
\usepackage{amssymb,amsmath}
\usepackage{polyglossia}
\setdefaultlanguage{spanish}
\usepackage[style=spanish]{csquotes}
\usepackage{pstricks-add}
\usepackage{tkz-euclide}
\usetkzobj{all}
\usepackage{pgf,tikz}
\usetikzlibrary{mindmap,trees,arrows}
\usepackage{siunitx}
\usepackage{xcolor}
\usepackage{booktabs}
\usepackage{marvosym}
\setbeamertemplate{caption}[numbered]
\usepackage{hyperref}
%-------------------------------------------------------------------------------------------------------
\def\talkclass{Presentación}
\def\talkcar{4TP}
\def\talkdate{\today}
\def\talkversion{}
\def\talktitle{La circunferencia}
\def\talksubtitle{Secciones cónicas como lugares geométricos}
\def\talkkeywords{circunferencia}
\def\talksubject{secciones cónicas}
\def\talkblog{https://hsigrist.github.io}
\def\talkpubpdf{https://github.com/hsigrist/LMLA/blob/master/3A/talks/circunferencia.pdf}
\def\talkcopyright{\myauthor}
\def\talkaffiliation{Liceo Mixto Los Andes}
\def\talkauthor{Hans Sigrist}
\def\talkgrade{Lic. \& Mag. Matemática}
\def\talkemail{hsigrist@liceomixto.cl}
\definecolor{links}{HTML}{000000}
\def\NN{\mathbb{N}}
\def\RR{\mathbb{R}}
\def\ZZ{\mathbb{Z}}
\def\QQ{\mathbb{Q}}
\def\II{\mathbb{I}}
\definecolor{bluu}{RGB}{7,29,66}
\newcommand{\framedhref}[2]{\href{#1}{\fcolorbox{bluu}{bluu}{\textcolor{white}{#2}}}}
\newtheorem{teorema}{Teorema}[section]
\newtheorem{lema}[teorema]{Lema}
\newtheorem{proposicion}[teorema]{Proposición}
\newtheorem{corolario}[teorema]{Corolario}
\newtheorem{definicion}[teorema]{Definición}
\newtheorem{ejemplo}[teorema]{Ejemplo}
\newtheorem{nota}[teorema]{Nota}
%-------------------------------------------------------------------------------------------------------
\hypersetup{pdfpagemode=FullScreen,colorlinks,linkcolor=,citecolor=black,urlcolor=links,pdftitle={pdftitle},pdfauthor={\talkauthor},pdfsubject={\talksubject},pdfkeywords={\talkkeywords}}
%-------------------------------------------------------------------------------------------------------
\setmainfont[Mapping={tex-text},Ligatures=TeX]{Linux Biolinum
    O}
\setsansfont[Mapping={tex-text},Ligatures=TeX]{Linux Libertine
    O}
\setmonofont[Mapping={tex-text},Numbers={OldStyle},Ligatures=Rare,Scale=0.8]{Pragmata
    Pro Mono}
%-------------------------------------------------------------------------------------------------------
\graphicspath{{/home/hsigrist/Dropbox/images/}}
\everymath{\displaystyle}
\AtBeginSection[]{\begin{frame}<beamer>\frametitle{Agenda}\tableofcontents[sectionstyle=show/hide,subsectionstyle=hide/show/hide,currentsection]\end{frame}\addtocounter{framenumber}{-1}}
%-------------------------------------------------------------------------------------------------------
\title{\talktitle}
\subtitle{\talksubtitle}
\author{\talkauthor}
\institute{\talkaffiliation}
\date{\footnotesize{\emph{\href{\talkblog}{\talkemail}}}}
%-------------------------------------------------------------------------------------------------------
\begin{document}
\begin{frame}
\titlepage
\end{frame}
%-------------------------------------------------------------------------------------------------------
\section{La circunferencia como lugar geométrico}
\begin{frame}
  \frametitle{Secciones cónicas}
  \begin{block}{Definición}
    
  \begin{minipage}[t]{.4\linewidth}
    
  \begin{tikzpicture}[parabola/.style={thick, red}, hyperbola/.style={thick, violet}, circle/.style={thick, blue}]
    \def\b{2}
    \def\h{2}
    \def\p{0.5}
    \pgfmathsetmacro{\rx}{\b/2}
    \pgfmathsetmacro{\ry}{\rx*\p}
    \pgfmathsetmacro{\ta}{90-atan2(\h,\ry)}
    \fill[gray!50]
    (0, \h) -- (\ta:\rx+0 and \ry) arc (\ta:180-\ta:\rx+0 and \ry) -- cycle;
    \fill[gray!75] coordinate (bottom) ellipse [x radius=\rx, y radius=\ry];
    \draw[dashed] (\ta:\rx+0 and \ry) arc (\ta:180-\ta:\rx+0 and \ry);
    \draw (0, \h) -- (\ta:\rx+0 and \ry) arc (\ta:-180-\ta:\rx+0 and \ry) -- coordinate[pos=.4](hypbend) cycle;
    \begin{scope}[rotate around={180:(0,\h)}]
      \fill[gray!50]
      (0, \h) -- (\ta:\rx+0 and \ry) arc (\ta:180-\ta:\rx+0 and \ry) -- cycle;
      \fill[gray!75] coordinate (top) ellipse [x radius=\rx, y radius=\ry];
      \draw (\ta:\rx+0 and \ry) arc (\ta:180-\ta:\rx+0 and \ry) (0, \h) -- coordinate[pos=.3](parabend) (\ta:\rx+0 and \ry) arc (\ta:-180-\ta:\rx+0 and \ry) -- cycle;
    \end{scope}
    %% Dots & Axis
    \fill (top) circle[radius=1pt] (0,\h) circle[radius=1pt] (bottom) circle[radius=1pt];
    \draw[dotted] (top) -- (bottom);
    \draw (top) -- ++(-90:\rx+0 and \ry);
    %% Parabola
    \path (top) +(50+50*\p:\rx+0 and \ry) coordinate (tmp) +(-50+50*\p:\rx+0 and \ry) coordinate (tmp2);
    \draw[parabola] (tmp) ..controls ++(-90:0) and ++(\ta+90:\p).. (parabend) node[left]{Parábola} ..controls ++(\ta-90:\p) and ++(-90:0).. (tmp2);
    %% Circle
    \def\pos{0.5}
    \draw[circle] (0,\pos*\h) ellipse [x radius=\pos*\rx, y radius=\pos*\ry] +(180:\pos*\rx) node[above left]{Círculo};
    %% Hyperbola
    \path (bottom) +(130-50*\p:\rx+0 and \ry) coordinate (tmp) +(230+50*\p:\rx+0 and \ry) coordinate (tmp2);
    \draw[hyperbola] (tmp) ..controls ++(90:0) and ++(90-\ta:0.6*\p).. (hypbend) node[left]{Hipérbola} ..controls ++(-\ta-90:0.6*\p) and ++(90:0).. (tmp2);
  \end{tikzpicture}
  \end{minipage}
  \begin{minipage}[b]{.55\linewidth}
      Considere $L$ una recta en una posición fija en el espacio y $A$ un punto
      tal que $A\in L$. Si se trazan infintas rectas $L_1$ que pasan por $A$ y
      forman un cierto ángulo $\alpha$ con $L$, se genera un \textbf{doble cono
        de revolución}.
      \vspace{1cm}
  \end{minipage}
    \end{block}
\end{frame}
%-------------------------------------------------------------------------------------------------------
\begin{frame}
  \frametitle{La circunferencia}
  \begin{block}{Definición}
    \begin{minipage}[b]{.45\linewidth}
      \begin{tikzpicture}[scale=.75]
        \tkzInit[xmax=5,ymax=5]
        \tkzDrawX \tkzDrawY
        \tkzDefPoint(3,3){O} \tkzDefPoint(4,4){P}
        \tkzCalcLength[cm](O,P) \tkzGetLength{rOP}
        \tkzDrawCircle[R,color=bluu](O,\rOP cm)
        \tkzDrawPoints(O)
        \tkzDrawPoints(P)
        \tkzLabelPoint[below](O){$O(h,k)$}
        \tkzLabelPoint[above right](P){$P(x,y)$}
        \tkzDrawSegment(O,P)
        \tkzLabelSegment(O,P){$r$}
        \tkzDefPoint(4,0){x}
        \tkzDefPoint(0,4){y}
        \tkzDrawSegment[style=dashed](P,x)
        \tkzDrawSegment[style=dashed](P,y)
        \tkzLabelPoint[below left](x){$x$}
        \tkzLabelPoint[left](y){$y$}
      \end{tikzpicture}
    \end{minipage}
    \begin{minipage}[b]{.45\linewidth}
      La circunferencia es el lugar geométrico de todos los puntos $P(x,y)$ del
      plano que se encuentran a una distancia determinada $r$ de un punto dado
      $O(h,k)$ de dicho plano.
      \begin{align*}
        \mathcal{C}(O,r)&=\left\{ P\in\mathcal{P}/OP=r \right\}
      \end{align*}
    \end{minipage}
  \end{block}
\end{frame}
%-------------------------------------------------------------------------------------------------------
\begin{frame}
  \frametitle{Ecuación principal de la circunferencia}
  \begin{block}{Definición}
    \begin{align*}
     (x-h)^2+(y-k)^2&=r^2 
    \end{align*}
    Esta expresión corresponde a la ecuación principal de la circunferencia, con
    $h$ y $k$ las coordenadas del centro y $r$ el radio. 
  \end{block}
\end{frame}
%-------------------------------------------------------------------------------------------------------
\begin{frame}
  \frametitle{Ejemplo}
  \begin{exampleblock}{Determinar la ecuación principal de la circunferencia
      cuyo centro es $(4,2)$ y su radio es $3$.}
\pause
    \begin{tikzpicture}[scale=.75]
        \tkzInit[xmax=9,ymax=5, ymin=-2]
        \tkzDrawX \tkzDrawY
        \tkzDefPoint(4,2){O} \tkzDefPoint(5.92,4.31){P}
        \tkzCalcLength[cm](O,P) \tkzGetLength{rOP}
        \tkzDrawCircle[R,color=green](O,\rOP cm)
        \tkzDrawPoints(O)
        \tkzDrawPoints(P)
        \tkzLabelPoint[below](O){$O(4,2)$}
        \tkzLabelPoint[above right](P){$P(x,y)$}
        \tkzDrawSegment(O,P)
        \tkzLabelSegment[left](O,P){$r=3$}
        \tkzDefPoint(5.92,0){x}
        \tkzDefPoint(0,4.31){y}
        \tkzDrawPoints(x)
        \tkzDrawPoints(y)
        \tkzDrawSegment[style=dashed](P,x)
        \tkzDrawSegment[style=dashed](P,y)
        \tkzLabelPoint[below left](x){$x$}
        \tkzLabelPoint[left](y){$y$}
        \tkzLabelCircle[R,right,text centered](O,\rOP cm)(-1){$(x-4)^2+(y-2)^2=9$}
      \end{tikzpicture}
  \end{exampleblock}
\end{frame}
%-------------------------------------------------------------------------------------------------------
\begin{frame}
  \frametitle{La circunferencia centrada en el origen}
  \begin{block}{Caso $h=k=0$: ecuación canónica de la circunferencia}
      \begin{tikzpicture}[scale=.75]
        \tkzInit[xmax=4,xmin=-4,ymax=4,ymin=-4]
        \tkzDrawX[noticks] \tkzDrawY[noticks]
        \tkzDefPoint(0,0){O} \tkzDefPoint(2.56,1.57){P}
        \tkzCalcLength[cm](O,P) \tkzGetLength{rOP}
        \tkzDrawCircle[R,color=bluu](O,\rOP cm)
        \tkzDrawPoints(O)
        \tkzDrawPoints(P)
        \tkzLabelPoint[below left](O){$O(0,0)$}
        \tkzLabelPoint[above right](P){$P(x,y)$}
        \tkzDrawSegment(O,P)
        \tkzLabelSegment(O,P){$r$}
        \tkzDefPoint(2.56,0){x}
        \tkzDefPoint(0,1.57){y}
        \tkzDrawPoints(x,y)
        \tkzDrawSegment[style=dashed](P,x)
        \tkzDrawSegment[style=dashed](P,y)
        \tkzLabelPoint[below left](x){$x$}
        \tkzLabelPoint[left](y){$y$}
        \tkzLabelCircle[R,right=50pt,text centered](O,\rOP cm)(40){$x^2+y^2=r^2$}
      \end{tikzpicture}
  \end{block}
\end{frame}
%-------------------------------------------------------------------------------------------------------
\begin{frame}
  \frametitle{Ecuación general de la circunferencia}
  \begin{block}{Definición}
    A partir de la ecuación principal de la circunferencia:
    \begin{align*}
      (x-h)^2+(y-k)^2&=r^2
    \end{align*}
    podemos obtener la denominada \textbf{ecuación general de la
      circunferencia}:
    \begin{align*}
      x^2+y^2+Dx+Ey+F&=0
    \end{align*}
  \end{block}
\end{frame}
%-------------------------------------------------------------------------------------------------------
\begin{frame}
  \frametitle{Ejemplo}
  \begin{exampleblock}{Determine los coeficientes $D,E$ y $F$ de la ecuación general de la
      circunferencia con centro en $(2,5)$ y radio $r=6$.}
    \begin{align*}
      h=2,k=5,r=6&\Rightarrow (x-2)^2+(y-5)^2=6\\
                 &\Rightarrow x^2+y^2-4x-10y-7=0\\
                 &\Rightarrow D=-4,E=-10\wedge F=-7.
    \end{align*}
  \end{exampleblock}
\end{frame}
%-------------------------------------------------------------------------------------------------------
\begin{frame}
  \frametitle{Problema}
  \begin{exampleblock}{Determine la ecuación de la circunferencia si uno de sus
      diámetros es el segmento que une los puntos  $A(-5,7)$ y $B(7,-3)$.}
    \pause
    \begin{align*}
      (h,k)&=\left(\frac{-5+7}{2},\frac{7-3}{2}\right)\\
      &=(1,2)\\
      \Rightarrow&r=\sqrt{(7-1)^2+(-3-2)^2}\\
       &=\sqrt{36+25}\\
      &=\sqrt{61}\\
      \Rightarrow&(x-1)^2+(y-2)^2=\left(\sqrt{61}\right)^2\\
      \Rightarrow&x^2+y^2-2x-4y-56=0\qed
    \end{align*}
  \end{exampleblock}
\end{frame}
%-------------------------------------------------------------------------------------------------------
\begin{frame}
  \frametitle{Parametrización de una circunferencia}
  \begin{block}{Depende de tres parámetros}
    De acuerdo a lo visto anteriormente, la ecuación de una circunferencia se
    puede escribir en su \alert{forma principal}:
    \begin{align*}
      (x-h)^2+(y-k)^2&=r^2
    \end{align*}
    o bien, en su \alert{forma general}
    \begin{align*}
      x^2+y^2+Dx+Ey+F&=0
    \end{align*}
    En ambos casos, la ecuación depende de \alert{tres parámetros}:
    \begin{align*}
      h,k,r\hspace{.5cm}\vee\hspace{.5cm} D,E,F
    \end{align*}
  \end{block}
\end{frame}
%-------------------------------------------------------------------------------------------------------
\begin{frame}
  \frametitle{Problema}
  \begin{exampleblock}{Encontrar la ecuación de la circunferencia que pasa por
      los puntos $A(1,0)$, $B(3,-2)$ y $C(1,-4)$}
    \pause

    \[
      \systeme*{1+0+D\cdot1+E\cdot0+F=0,9+4+D\cdot3+E\cdot(-2)+F=0,1+16+D\cdot1+E\cdot(-4)+F=0}
      \systeme*{1+D+F=0,13+3D-2E+F=0,17+D-4E+F=0}
      \systeme*{D=-2,E=4,F=1}
    \]
    \begin{align*}
      \therefore x^2+y^2-2x+4y+1&=0\qed
    \end{align*}
  \end{exampleblock}
\end{frame}
%-------------------------------------------------------------------------------------------------------

\begin{frame}
  \frametitle{Problema}
  \begin{exampleblock}{Encontrar la ecuación de la circunferencia que pasa por
      los puntos $A(-2,5)$, $B(3,2)$ y $C(0,0)$}
  \end{exampleblock}
\end{frame}
%-------------------------------------------------------------------------------------------------------
\begin{frame}[c]\frametitle{Apéndice}
\centering\decofourleft\quad\decofourright

\textbf{\emph {¡Carpe diem!}}

Una copia del presente trabajo, se encuentra en el enlace \framedhref{\talkpubpdf}{\talktitle}.
\end{frame}
\end{document}
%-------------------------------------------------------------------------------------------------------
%!TEX program = xelatex
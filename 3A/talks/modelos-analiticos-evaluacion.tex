\documentclass[12pt,spanish,x11names]{beamer}
%-------------------------------------------------------------------------------------------------------
% \usepackage{pgfpages}
% \setbeameroption{hide notes}
% \setbeameroption{show notes}
% \setbeameroption{show notes on second screen=right}
\usetheme{Hytex}
\setbeamertemplate{navigation symbols}{}
\usecolortheme[RGB={7,29,66}]{structure}
\usepackage{tcolorbox}
\usepackage{fourier}
\usepackage{float}
\usepackage{fontspec}
\usepackage{graphicx}
\usepackage{amssymb,amsmath}
\usepackage{polyglossia}
\setdefaultlanguage{spanish}
\usepackage[style=spanish]{csquotes}
\usepackage{pstricks-add}
\usepackage{tkz-euclide}
\usetkzobj{all}
\usepackage{pgf,tikz}
\usetikzlibrary{mindmap,trees,arrows}
\usepackage{siunitx}
\usepackage{xcolor}
\usepackage{booktabs}
\usepackage{marvosym}
\setbeamertemplate{caption}[numbered]
\usepackage{hyperref}
%-------------------------------------------------------------------------------------------------------
\def\talkclass{Presentación}
\def\talkcar{3HC}
\def\talkdate{\today}
\def\talkversion{}
\def\talktitle{Modelos Analíticos}
\def\talksubtitle{Evaluación}
\def\talkkeywords{pendiente, razones trigonométricas, distancia entre dos
  puntos, transversal de gravedad, punto medio}
\def\talksubject{geometría analítica}
\def\talkblog{https://hsigrist.github.io}
\def\talkpubpdf{https://www.dropbox.com/s/tuhjx03333fl5vu/todo-es-bullying.pdf?dl=0}
\def\talkcopyright{\myauthor}
\def\talkaffiliation{Liceo Mixto Los Andes}
\def\talkauthor{Hans Sigrist}
\def\talkgrade{Lic. \& Mag. Matemática}
\def\talkemail{hsigrist@liceomixto.cl}
\definecolor{links}{HTML}{000000}
\def\NN{\mathbb{N}}
\def\RR{\mathbb{R}}
\def\ZZ{\mathbb{Z}}
\def\QQ{\mathbb{Q}}
\def\II{\mathbb{I}}
\definecolor{bluu}{RGB}{7,29,66}
\newcommand{\framedhref}[2]{\href{#1}{\fcolorbox{bluu}{bluu}{\textcolor{white}{#2}}}}
\newtheorem{teorema}{Teorema}[section]
\newtheorem{lema}[teorema]{Lema}
\newtheorem{proposicion}[teorema]{Proposición}
\newtheorem{corolario}[teorema]{Corolario}
\newtheorem{definicion}[teorema]{Definición}
\newtheorem{ejemplo}[teorema]{Ejemplo}
\newtheorem{nota}[teorema]{Nota}
%-------------------------------------------------------------------------------------------------------
\hypersetup{pdfpagemode=FullScreen,colorlinks,linkcolor=,citecolor=black,urlcolor=links,pdftitle={pdftitle},pdfauthor={\talkauthor},pdfsubject={\talksubject},pdfkeywords={\talkkeywords}}
%-------------------------------------------------------------------------------------------------------
\setmainfont[Mapping={tex-text},Numbers={OldStyle},Ligatures=TeX]{Linux Biolinum
    O}
\setsansfont[Mapping={tex-text},Numbers={OldStyle},Ligatures=TeX]{Linux Libertine
    O}
\setmonofont[Mapping={tex-text},Numbers={OldStyle},Ligatures=Rare,Scale=0.8]{Pragmata
    Pro Mono}
%-------------------------------------------------------------------------------------------------------
\graphicspath{{/home/hsigrist/Dropbox/images/}}
\everymath{\displaystyle}
\AtBeginSection[]{\begin{frame}<beamer>\frametitle{Agenda}\tableofcontents[sectionstyle=show/hide,subsectionstyle=hide/show/hide,currentsection]\end{frame}\addtocounter{framenumber}{-1}}
%-------------------------------------------------------------------------------------------------------
\title{\talktitle}
\subtitle{\talksubtitle}
\author{\talkauthor}
\institute{\talkaffiliation}
\date{\footnotesize{\emph{\href{\talkblog}{\talkemail}}}}
%-------------------------------------------------------------------------------------------------------
\begin{document}
\begin{frame}
\titlepage
\end{frame}
%-------------------------------------------------------------------------------------------------------
% \section{Elementos del triángulo}
\begin{frame}
  \frametitle{Transversal de gravedad}
  \begin{minipage}[t]{.45\linewidth}
     \begin{figure}[h]
      \centering
      \begin{tikzpicture}
        \tkzInit[xmax=4, ymax=5]
        \tkzAxeXY
        \tkzGrid
        \tkzDefPoint(1,0){A}
        \tkzDefPoint(4,1){B}
        \tkzDefPoint(2,5){C}
        \tkzDrawPolygon(A,B,C)
        \tkzCentroid(A,B,C)\tkzGetPoint{G}
        \tkzDrawLines[add = 0 and 3/4](A,G)
        \tkzDefMidPoint(B,C) \tkzGetPoint{D}
        \tkzDrawPoint(D)
        \tkzLabelPoint[below](D){$D$}
        \tkzLabelPoint[above left](A){$A$}
        \tkzLabelPoint[right](B){$B$}
        \tkzLabelPoint[above](C){$C$}
        \tkzLabelLine[pos=1.8,above](A,G){$t_A$}
      \end{tikzpicture}
    \end{figure}
  \end{minipage}
  \begin{minipage}[t]{.25\linewidth}
    \begin{enumerate}
    \item $A(\ \ ,\ \ )=$
    \item $B(\ \ ,\ \ )=$
    \item $C(\ \ ,\ \ )=$
    \item $D(\ \ ,\ \ )=$
    \item $d(A,B)=$
    \item $d(B,C)=$
    \item $d(C,A)=$
    \item $d(A,D)=$
    \item $d(D,B)=$
    \item $d(D,C)=$
    \end{enumerate}
  \end{minipage}
  \begin{minipage}[t]{.25\linewidth}
    \begin{enumerate}
    \item (ec. principal) $t_A=$
    \item (ec. general) $t_A=$
    \item $P_{\triangle ABC}=$
    \item $A_{\triangle ABC}=$
    \end{enumerate}
  \end{minipage}
\end{frame}
%-------------------------------------------------------------------------------------------------------
\end{document}
%-------------------------------------------------------------------------------------------------------
% !TEX program = xelatex
\documentclass[12pt,addpoints,x11names]{exam}
%____________________________________________________________________________________
\usepackage{pgf,tikz}
\usepackage{tkz-euclide}
\usepackage{fourier}
\usepackage{fontspec}
\usepackage{graphicx}
\usepackage{amssymb,amsmath}
\usepackage{polyglossia}
\setdefaultlanguage{spanish}
\usetikzlibrary{arrows}
\usepackage{siunitx}
\usepackage{xcolor}
\usepackage{multicol}
\usepackage{hyperref}
\usetkzobj{all}
\usepackage[left=1.5cm,right=1.5cm,top=2.5cm,bottom=1.5cm,paperheight=33cm]{geometry}
%____________________________________________________________________________________
\setromanfont[Mapping={tex-text}]{Linux Biolinum O}
\setsansfont[Mapping={tex-text}]{Cantarell}
\setmonofont[Mapping={tex-text},Scale=0.8]{Pragmata Pro}
%____________________________________________________________________________________
\graphicspath{{/home/hsigrist/Dropbox/images/}}
\everymath{\displaystyle}
\def\NN{\mathbb{N}}
\def\RR{\mathbb{R}}
\def\QQ{\mathbb{Q}}
\def\ZZ{\mathbb{Z}}
\def\II{\mathbb{I}}
\newcommand{\framedhref}[2]{\href{#1}{\fcolorbox{SlateGray1}{SlateGray1}{#2}}}
%____________________________________________________________________________________
\everymath{\displaystyle}
\extraheadheight{1.5in}
\extrafootheight{.3in}
%____________________________________________________________________________________
\pagestyle{headandfoot}
\firstpageheader{\includegraphics[scale=.7]{logolmlabw}}
{\Large\textbf{Evaluación Sumativa 1}\\
  Números Complejos\\
  Tercero Medio TP\\
  marzo 2018}
{\begin{tikzpicture}
    \draw (0,0) rectangle (4,4);
    \draw (2.5,0) rectangle (4,1.5);
 \end{tikzpicture}}
\runningheader{Evaluación 1 - Números Complejos / Tercero Medio TP}{}{Dpto. Matemática - LMLA $\cdot$ 2018, pág. \thepage}
\footer{}{}{}
%____________________________________________________________________________________
\def\mytitle{Evaluación Sumativa 1 Números complejos}
\def\mykeywords{imaginarios, potencias imaginarias, conjugado, módulo}
\def\mysubject{números complejos}
\def\myauthor{Prof. Hans Sigrist}
\definecolor{links}{HTML}{000000}
%____________________________________________________________________________________
\hypersetup{%
  colorlinks,%
  linkcolor=,%
  citecolor=black,%
  urlcolor=links,%
  pdftitle={\mytitle},%
  pdfauthor={\myauthor},%
  pdfsubject={\mysubject},%
  pdfkeywords={\mykeywords}%
}
%____________________________________________________________________________________
% \printanswers
%____________________________________________________________________________________
\newcommand*\Matching[1]{%
  \ifprintanswers%
  \textbf{#1}%
  \else%
  \rule{2.1in}{0.5pt}%
  \fi%
}
%____________________________________________________________________________________
\newlength\matchlena
\newlength\matchlenb
\settowidth\matchlena{\rule{2.1in}{0pt}}
\newcommand\MatchQuestion[2]{%
  \setlength\matchlenb{\linewidth}%
  \addtolength\matchlenb{-\matchlena}%
  \parbox[t]{\matchlena}{\Matching{#1}}\enspace\parbox[t]{\matchlenb}{#2}%
}
%____________________________________________________________________________________
\begin{document}
%____________________________________________________________________________________
\addpoints
\renewcommand{\solutiontitle}{\noindent\textbf{Solución:}\enspace}
\pointname{ puntos}
\pointpoints{ punto}{ puntos}
\bracketedpoints
\boxedpoints
\CorrectChoiceEmphasis{\color{red}\bfseries}
\renewcommand{\choicelabel}{\thechoice)}
\chqword{Pregunta}
\chpword{Puntos}
\chbpword{Bonus}
\chsword{Total}\hqword{Pregunta}
\hpword{Puntos}
\cellwidth{3mm}
\setlength\answerskip{2ex}
\setlength\answerlinelength{2in}
%____________________________________________________________________________________
\fullwidth{\noindent\large\textbf{Nombre:}\enspace\makebox[5.2in]{\hrulefill} \textbf{Curso:}\enspace\makebox[0.5in]{\hrulefill}}
\fullwidth{%
  \begin{description}
  \item[Objetivos:] Calcular raíces imaginarias. Calcular potencias de números imaginarios. Obtener el conjugado y el módulo de números complejos. Realizar operatoria combinada en números complejos.
  \item[Instrucciones:] \textbf{Tiempo de duración de la Evaluación 80 minutos}. Lea atentamente las situaciones plantaedas, así como las intrucciones y conteste ofreciendo un desarrollo según corresponda. Evite borrones, use solamente lápiz grafito, no se permite el uso de calculadora ni celulares.
  \item[Puntaje:] Puntaje total: \numpoints\  puntos 7.0 y 60\%: \pgfmathparse{div(\numpoints*60,100)}\pgfmathresult\  puntos 4.0.
  \end{description}
}
%----------------------------------------------------------------------------------------------------------------------------------------
\fullwidth{\noindent{\large \textbf{I. Raíces imaginarias}}}
\begin{questions}
 \question[2] $\sqrt{-169}=$
   \begin{solutionbox}{.5in}
    $\sqrt{-169}=\sqrt{169\cdot(-1)}=\sqrt{169}\cdot\sqrt{-1}=13i$
  \end{solutionbox}
\question[2] $\sqrt{-100}=$
   \begin{solutionbox}{.5in}
    $\sqrt{-100}=\sqrt{100\cdot(-1)}=\sqrt{100}\cdot\sqrt{-1}=10i$
  \end{solutionbox}
\question[2] $\sqrt{-81}=$
   \begin{solutionbox}{1in}
    $\sqrt{-81}=\sqrt{81\cdot(-1)}=\sqrt{81}\cdot\sqrt{-1}=9i$
  \end{solutionbox}
\question[2] $\sqrt{-144}=$
   \begin{solutionbox}{1in}
    $\sqrt{-144}=\sqrt{144\cdot(-1)}=\sqrt{144}\cdot\sqrt{-1}=12i$
  \end{solutionbox}
\question[2] $\sqrt{-289}=$
  \begin{solutionbox}{1in}
    $\sqrt{-289}=\sqrt{289\cdot(-1)}=\sqrt{289}\cdot\sqrt{-1}=17i$
  \end{solutionbox}

  \pagebreak
%----------------------------------------------------------------------------------------------------------------------------------------
\fullwidth{\noindent{\large \textbf{II. Cálculo de potencias imaginarias}}}
\question[3] $i^{32}+2i^{29}=$
  \begin{solutionbox}{1.5in}
    $i^{32}+2i^{29}=1+2i$
  \end{solutionbox}
 \question[3] $i^{121}-3i^{422}+2i^{329}=$
   \begin{solutionbox}{1.5in}
    $i^{121}-3i^{422}+2i^{329}=i-3\cdot(-1)+2\cdot(i)=i+3+2i=3+3i$
  \end{solutionbox}
\question[3] $3i^{4}+12i^{3}-23i^{2}=$
   \begin{solutionbox}{1.5in}
     $3i^{4}+12i^{3}-23i^{2}=3\cdot1+12\cdot(-i)-23\cdot(-1)=3-12i+23=26-12i$
  \end{solutionbox}
\question[3] $i^{11}+i^{12}+i^{13}+i^{14}+i^{15}=$
  \begin{solutionbox}{1.5in}
     $i^{11}+i^{12}+i^{13}+i^{14}+i^{15}=-i+1+i+(-1)+(-i)=-i+1-1-i=-2i$
  \end{solutionbox}
%----------------------------------------------------------------------------------------------------------------------------------------












\end{questions}
\end{document}

%%% Local Variables:
%%% mode: latex
%%% TeX-master: t
%%% End:

\documentclass[12pt,addpoints,x11names]{exam}
%____________________________________________________________________________________
\usepackage{float}
\usepackage{pgf,tikz}
\usepackage{tkz-euclide}
\usepackage{fourier}
\usepackage{fontspec}
\usepackage{graphicx}
\usepackage{amssymb,amsmath}
\usepackage{polyglossia}
\setdefaultlanguage{spanish}
\usetikzlibrary{arrows}
\usepackage{siunitx}
\usepackage{xcolor}
\usepackage{multicol}
\usepackage{hyperref}
\usetkzobj{all}
\usepackage[left=1.5cm,right=1.5cm,top=2.5cm,bottom=1.5cm,paperheight=33cm]{geometry}
%____________________________________________________________________________________
\setmainfont[Mapping={tex-text}]{Linux Biolinum O}
\setsansfont[Mapping={tex-text}]{Linux Libertine O}
\setmonofont[Mapping={tex-text},Scale=0.8]{Pragmata Pro}
%____________________________________________________________________________________
\graphicspath{{/home/hsigrist/Dropbox/images/}}
\everymath{\displaystyle}
\def\NN{\mathbb{N}}
\def\RR{\mathbb{R}}
\def\QQ{\mathbb{Q}}
\def\ZZ{\mathbb{Z}}
\def\II{\mathbb{I}}
\newcommand{\framedhref}[2]{\href{#1}{\fcolorbox{SlateGray1}{SlateGray1}{#2}}}
%____________________________________________________________________________________
\everymath{\displaystyle}
\extraheadheight{1.5in}
\extrafootheight{.3in}
%____________________________________________________________________________________
\pagestyle{headandfoot}
\firstpageheader{\includegraphics[scale=.7]{logolmlabw}}
{\Large\textbf{Evaluación Sumativa 2}\\
  Números Complejos\\
  Tercero Medio TP\\
  abril 2018}
{\begin{tikzpicture}
    \draw (0,0) rectangle (4,4);
    \draw (2.5,0) rectangle (4,1.5);
 \end{tikzpicture}}
\runningheader{Evaluación 2 - Números Complejos / Tercero Medio TP}{}{Dpto. Matemática - LMLA $\cdot$ 2018, pág. \thepage}
\footer{}{}{}
%____________________________________________________________________________________
\def\mytitle{Evaluación Sumativa 2 Números complejos}
\def\mykeywords{imaginarios, potencias imaginarias, conjugado, módulo}
\def\mysubject{números complejos}
\def\myauthor{Prof. Hans Sigrist}
\definecolor{links}{HTML}{000000}
%____________________________________________________________________________________
\hypersetup{%
  colorlinks,%
  linkcolor=,%
  citecolor=black,%
  urlcolor=links,%
  pdftitle={\mytitle},%
  pdfauthor={\myauthor},%
  pdfsubject={\mysubject},%
  pdfkeywords={\mykeywords}%
}
%____________________________________________________________________________________
% \printanswers
%____________________________________________________________________________________
\newcommand*\Matching[1]{%
  \ifprintanswers%
  \textbf{#1}%
  \else%
  \rule{2.1in}{0.5pt}%
  \fi%
}
%____________________________________________________________________________________
\newcommand{\midmatch}{\hspace{0.75in}\underline{\hspace{0.5in}}}
%____________________________________________________________________________________
\newlength\matchlena
\newlength\matchlenb
\settowidth\matchlena{\rule{2.1in}{0pt}}
\newcommand\MatchQuestion[2]{%
  \setlength\matchlenb{\linewidth}%
  \addtolength\matchlenb{-\matchlena}%
  \parbox[t]{\matchlena}{\Matching{#1}}\enspace\parbox[t]{\matchlenb}{#2}%
}
%____________________________________________________________________________________
\newcommand*{\TrueFalse}[1]{%
\ifprintanswers
    \ifthenelse{\equal{#1}{T}}{%
        \textbf{VERDADERO}\hspace*{14pt}F
    }{
        V\hspace*{14pt}\textbf{FALSO}
    }
\else
    {V}\hspace*{20pt}F
\fi
} 
\newlength\TFlengthA
\newlength\TFlengthB
\settowidth\TFlengthA{\hspace*{.7in}}
\newcommand\TFQuestion[2]{%
    \setlength\TFlengthB{\linewidth}
    \addtolength\TFlengthB{-\TFlengthA}
    \parbox[t]{\TFlengthA}{\TrueFalse{#1}}\parbox[t]{\TFlengthB}{#2}}
%____________________________________________________________________________________
\begin{document}
%____________________________________________________________________________________
\addpoints
\renewcommand{\solutiontitle}{\noindent\textbf{Solución:}\enspace}
\pointname{ puntos}
\pointpoints{ punto}{ puntos}
\bracketedpoints
\boxedpoints
\CorrectChoiceEmphasis{\color{red}\bfseries}
\renewcommand{\choicelabel}{\thechoice)}
\chqword{Pregunta}
\chpword{Puntos}
\chbpword{Bonus}
\chsword{Total}\hqword{Pregunta}
\hpword{Puntos}
\cellwidth{3mm}
\setlength\answerskip{2ex}
\setlength\answerlinelength{2in}
%____________________________________________________________________________________
\fullwidth{\noindent\large\textbf{Nombre:}\enspace\makebox[5.2in]{\hrulefill} \textbf{Curso:}\enspace\makebox[0.5in]{\hrulefill}}
\fullwidth{%
  \begin{description}
  \item[Objetivos:] Obtener el conjugado y el módulo de números complejos. Realizar operatoria combinada en números complejos.
  \item[Instrucciones:] \textbf{Tiempo de duración de la Evaluación 80 minutos}. Lea atentamente las situaciones plantaedas, así como las intrucciones y conteste ofreciendo un desarrollo según corresponda. Evite borrones, use solamente lápiz grafito, no se permite el uso de calculadora ni celulares.
  \item[Puntaje:] Puntaje total: \numpoints\  puntos 7.0 y 60\%: \pgfmathparse{div(\numpoints*60,100)}\pgfmathresult\  puntos 4.0.
  \end{description}
}
%____________________________________________________________________________________
\begin{questions}
\fullwidth{\noindent{\large \textbf{I. Formas canónica, vectorial y
      representación cartesiana de números complejos.}}}

\fullwidth{Considere los números complejos representados en el
  plano cartesiano siguiente y a partir de ellos responda encerrando en un
  círculo \textbf{V} o bien \textbf{F} la afirmación correcta:}

\begin{minipage}[c]{.55\linewidth}
  \begin{figure}[H]
    \centering
    \begin{tikzpicture}[scale=.5]
      \tkzInit[xmax=7,ymax=7,xmin=-7,ymin=-7]
      \tkzAxeXY
      \tkzGrid
      \tkzDrawX[noticks]
      \tkzDrawY[noticks]
      \tkzDefPoint(0,0){O}
      \tkzDefPoint(4,2){A}
      \tkzDefPoint(-2,6){B}
      \tkzDefPoint(4,-5){C}
      \tkzDefPoint(-6,-6){D}
      \tkzDrawVectors(O,A O,B O,C O,D)
      \tkzLabelSegment[right=10pt](O,A){$\vec{z}$}
      \tkzLabelSegment[left=10pt](O,B){$\vec{w}$}
      \tkzLabelSegment[above=3pt](O,C){$\vec{u}$}
      \tkzLabelSegment[above=3pt](O,D){$\vec{v}$}
    \end{tikzpicture}
  \end{figure}
\end{minipage}

\begin{minipage}[c]{.48\linewidth}
  \question[2]\TFQuestion{T}{$u+z=8-3i$}
  \question[2]\TFQuestion{T}{$2z=8+4i$}
  \question[2]\TFQuestion{T}{$z=(2,-4)$}
  \question[2]\TFQuestion{T}{$-3u=12+15i$}
  \question[2]\TFQuestion{T}{$w=(-2,6)$}
  \question[2]\TFQuestion{T}{$v=-6-6i$}
  \question[2]\TFQuestion{F}{$u=4+5i$}
\end{minipage}


\question[10] Complete la siguiente tabla (2 pts. c/u):
  \begin{center}
    \begin{tabular}[t]{|>{\centering\arraybackslash}m{4cm}|>{\centering\arraybackslash}m{4cm}|}
      \hline
      \textbf{Forma canónica}&\textbf{Forma vectorial}\\
      \hline
      $u=12-12i$ & \\[.7cm]
      \hline
                             & $v=(-8,8)$\\[.7cm]
      \hline
      $z=-3-9i$ & \\[.7cm]
      \hline
                             & $w=(-1,2)$\\[.7cm]
      \hline
      $t=\frac{1}{2}+\frac{1}{2}i$ & \\[.7cm]
      \hline
    \end{tabular}
  \end{center}
  \pagebreak
%----------------------------------------------------------------------------------------------------------------------------------------
\fullwidth{\noindent{\large \textbf{II. Operatoria combinada en números complejos}}}

En los siguientes ejercicios, considere:
\begin{eqnarray*}
  z_1&=&1-i\\
  z_2&=&4-3i\\
  z_3&=&2+6i\\
  z_4&=&1+7i
\end{eqnarray*}
 
\question[3] $z_3+z_1=$
  \begin{solutionbox}{1.2in}
     $z_3+z_1=1-7i+1+i=2-6i$
  \end{solutionbox}
\question[3] $z_1-2\cdot z_4+4\cdot z_2=$
  \begin{solutionbox}{1.2in}
     $z_1-2\cdot z_4+4\cdot z_2=1+i-2\cdot(1-7i)+4\cdot(4+3i)=1+i-2+14i+16+12i=15+27i$
  \end{solutionbox}
\question[3] $z_3\cdot z_4=$
  \begin{solutionbox}{1.2in}
     $z_3\cdot z_4=(2-6i)\cdot(1-7i)=2-14i-6i-42=-40-20i$
  \end{solutionbox}
\question[3] $\frac{z_1}{z_2}=$
  \begin{solutionbox}{1.2in}
     $\frac{z_1}{z_2}=\frac{1+i}{4+3i}=\frac{(1+i)(4-3i)}{(4-3i)(4-3i)}=\frac{16-12i-12i-9}{16+9}=\frac{7-24i}{25}$
  \end{solutionbox}
%____________________________________________________________________________________
\pagebreak

\fullwidth{\noindent{\large \textbf{III. Cálculo de módulo y conjugado de números complejos}}}
En los siguientes ejercicios, considere:

\begin{eqnarray*}
  z_5&=&2-3i\\
  z_6&=&1+i\\
  z_7&=&2-i\\
  z_8&=&3-4i
\end{eqnarray*}

\question[3] $\overline{z_8}+\overline{z_7}=$
   \begin{solutionbox}{1.3in}
     $\overline{z_8}+\overline{z_7}=3-4i+2-i=5-5i$
  \end{solutionbox}
\question[3] $\overline{z_7}-3\cdot\overline{z_5}=$
   \begin{solutionbox}{1.3in}
     $\overline{z_7}-3\cdot\overline{z_5}=2-i-3\cdot(2-3i)=2-i-6+9i=-4+8i$
  \end{solutionbox}
\question[3] $|z_5|+|z_6|=$
   \begin{solutionbox}{1.3in}
     $|z_5|+|z_6|=\sqrt{2^{2}+3^{2}}+\sqrt{1^{2}+1^{2}}=\sqrt{4+9}+\sqrt{1+1}=\sqrt{13}+\sqrt{2}$
  \end{solutionbox}
\question[3] $|z_8|-|z_7|=$
  \begin{solutionbox}{1.3in}
     $|z_8|-|z_7|=\sqrt{3^{2}+4^{2}}-\sqrt{2^{2}+1^{2}}=\sqrt{9+16}-\sqrt{4+1}=\sqrt{25}-\sqrt{5}=5-\sqrt{5}$
  \end{solutionbox}
\end{questions}
%____________________________________________________________________________________
\end{document}

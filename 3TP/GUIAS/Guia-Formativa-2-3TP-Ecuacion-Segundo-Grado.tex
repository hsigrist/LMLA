\documentclass[twocolumns,12pt,addpoints,x11names]{exam}
%--------------------------------------------------------------------------------------------------------------------------
\usepackage{pgf,tikz}
\usepackage{multicol}
\usepackage{tkz-euclide}
\usepackage{fourier}
\usepackage{fontspec}
\usepackage{multicol}
\usepackage{graphicx}
\usepackage{amssymb,amsmath}
\usepackage{polyglossia}
\setdefaultlanguage{spanish}
\usetikzlibrary{arrows}
\usepackage{siunitx}
\usepackage{xcolor}
\usepackage{multicol}
\usepackage{hyperref}
\usetkzobj{all}
\usepackage[left=1.5cm,right=1.5cm,top=2.5cm,bottom=1.5cm,paperheight=33cm]{geometry}
%----------------------------------------------------------------------------------------------------------------------------------------
\setmainfont[Mapping={tex-text}]{Linux Biolinum O}
\setsansfont[Mapping={tex-text}]{Linux Libertine O}
\setmonofont[Mapping={tex-text},Scale=0.8]{Pragmata Pro}
%----------------------------------------------------------------------------------------------------------------------------------------
\graphicspath{{/home/hsigrist/Dropbox/images/}}
\everymath{\displaystyle}
\def\NN{\mathbb{N}}
\def\RR{\mathbb{R}}
\def\QQ{\mathbb{Q}}
\def\ZZ{\mathbb{Z}}
\def\II{\mathbb{I}}
\newcommand{\framedhref}[2]{\href{#1}{\fcolorbox{SlateGray1}{SlateGray1}{#2}}}
%----------------------------------------------------------------------------------------------------------------------------------------
\everymath{\displaystyle}
\extraheadheight{1in}
\extrafootheight{.75in}
%----------------------------------------------------------------------------------------------------------------------------------------
\pagestyle{headandfoot}
\firstpageheader{\includegraphics[scale=.56]{logolmlabw}}
{\Large\textbf{Guía Formativa 2}\\
  Ecuaciones de segundo grado\\
  Tercero Medio TP\\
  mayo 2018}
{}
\runningheader{Guía 2 - Ecuaciones de segundo grado / Tercero Medio TP}{}{Dpto. Matemática - LMLA $\cdot$ 2018, pág. \thepage}
\footer{}{}{}
%----------------------------------------------------------------------------------------------------------------------------------------
\def\mytitle{sxsx}
\def\mykeywords{xs}
\def\mysubject{xs}
\def\myauthor{Prof. Hans Sigrist}
\definecolor{links}{HTML}{000000}
%----------------------------------------------------------------------------------------------------------------------------------------
\hypersetup{%
  colorlinks,%
  linkcolor=,%
  citecolor=black,%
  urlcolor=links,%
  pdftitle={\mytitle},%
  pdfauthor={\myauthor},%
  pdfsubject={\mysubject},%
  pdfkeywords={\mykeywords}%
}
%----------------------------------------------------------------------------------------------------------------------------------------
% \printanswers
%----------------------------------------------------------------------------------------------------------------------------------------
\newcommand*\Matching[1]{%
  \ifprintanswers%
  \textbf{#1}%
  \else%
  \rule{2.1in}{0.5pt}%
  \fi%
}
%----------------------------------------------------------------------------------------------------------------------------------------
\newlength\matchlena
\newlength\matchlenb
\settowidth\matchlena{\rule{2.1in}{0pt}}
\newcommand\MatchQuestion[2]{%
  \setlength\matchlenb{\linewidth}%
  \addtolength\matchlenb{-\matchlena}%
  \parbox[t]{\matchlena}{\Matching{#1}}\enspace\parbox[t]{\matchlenb}{#2}%
}
%----------------------------------------------------------------------------------------------------------------------------------------
\begin{document}
%----------------------------------------------------------------------------------------------------------------------------------------
\addpoints
\renewcommand{\solutiontitle}{\noindent\textbf{Solución:}\enspace}
\pointname{ puntos}
\pointpoints{ punto}{ puntos}
\bracketedpoints
\boxedpoints
\CorrectChoiceEmphasis{\color{red}\bfseries}
\renewcommand{\choicelabel}{\thechoice)}
\chqword{Pregunta}
\chpword{Puntos}
\chbpword{Bonus}
\chsword{Total}\hqword{Pregunta}
\hpword{Puntos}
\cellwidth{3mm}
\setlength\answerskip{2ex}
\setlength\answerlinelength{2in}
%----------------------------------------------------------------------------------------------------------------------------------------
\fullwidth{\noindent\large\textbf{Nombre:}\enspace\makebox[5.2in]{\hrulefill} \textbf{Curso:}\enspace\makebox[0.5in]{\hrulefill}}
\fullwidth{%
  \begin{description}
  \item[Objetivos:] Resolver ecuaciones de segundo grado utilizando diversos métodos. Analizar la naturaleza y características de sus soluciones.
  \end{description}
}
%----------------------------------------------------------------------------------------------------------------------------------------
\begin{multicols}{2}
\begin{questions}
\fullwidth{\noindent{\large \textbf{I. Ecuaciones de segundo grado incompletas de la forma $ax^{2}+c=0$.}}}
\fullwidth{Resuelva las siguientes ecuaciones incompletas.}
  \question $4x^{2}-16=0$
  \question $3x^{2}-27=0$
  \question $2x^{2}-10=0$
  \question $5x^{2}-20=0$
  \question $3x^{2}-13=0$
  \question $6x^{2}-28=0$
  \question $7x^{2}-13=0$
  \question $11x^{2}-23=0$
  \question $-6x^{2}+12=0$
  \question $-4x^{2}+256=0$
  \question $6561-81x^{2}=0$
  \question $1225x^{2}-25=0$
  \question $36x^{2}=1296$
  \question $2x^{2}+32=0$
  \question $4x^{2}+500=0$
  \question $16x^{2}+256=0$
  \question $10x^{2}+1000=0$
%----------------------------------------------------------------------------------------------------------------------------------------
\fullwidth{\noindent{\large \textbf{II. Ecuaciones de segundo grado incompletas de la forma $ax^{2}+bx=0$.}}}
\fullwidth{Resuelva las siguientes ecuaciones incompletas.}
  \question $x^{2}-9x=0$
  \question $x^{2}-16x=0$
  \question $x^{2}-32x=0$
  \question $x^{2}+21x=0$
  \question $2x^{2}+4x=0$
  \question $4x^{2}+18x=0$
  \question $-2x^{2}-4x=0$
  \question $5x^{2}-7x=0$
  \question $3x^{2}-9x=0$
  \question $7x^{2}+12x=0$
  \question $8x^{2}-24x=0$
  \question $9x^{2}-27x=0$
  \question $7x^{2}-21x=0$
  \question $6x^{2}-42x=0$
  \question $-3x^{2}+33x=0$
  \question $-4x^{2}+52x=0$
  \question $-6x^{2}+102x=0$
%----------------------------------------------------------------------------------------------------------------------------------------
\fullwidth{\noindent{\large \textbf{III. Ecuaciones de segundo grado completas de la forma $ax^{2}+bx+c=0$.}}}
\fullwidth{Resuelva las siguientes ecuaciones completas \textbf{usando factorización}.}
  \question $x^{2}+8x+15=0$
  \question $x^{2}+10x+21=0$
  \question $x^{2}+12x+27=0$
  \question $x^{2}+11x+24=0$
  \question $x^{2}+14x+24=0$
  \question $x^{2}+17x+42=0$
  \question $x^{2}+19x+65=0$
  \question $x^{2}+25x+66=0$
  \question $x^{2}-20x+1=0$
  \question $x^{2}-12x+27=0$
  \question $x^{2}-4x-21=0$
  \question $x^{2}+x-42=0$
  \question $x^{2}-5x+6=0$
  \question $x^{2}-4x+4=0$
  \question $x^{2}-5x-24=0$
  \question $x^{2}-3x+2=0$
  \question $x^{2}+x-2=0$
  \question $x^{2}-x-2=0$
%----------------------------------------------------------------------------------------------------------------------------------------
\fullwidth{\noindent{\large \textbf{IV. Ecuaciones de segundo grado completas de la forma $ax^{2}+bx+c=0$ mediante fórmula.}}}
\fullwidth{Resuelva las siguientes ecuaciones completas \textbf{usando fórmula}.}
  \question $x^{2}+8x+15=0$
  \question $x^{2}+7x+3=0$
  \question $x^{2}-2x-6=0$
  \question $x^{2}-4x-3=0$
  \question $x^{2}+6x+7=0$
  \question $x^{2}-4x+1=0$
  \question $2x^{2}-2x-3=0$
  \question $x^{2}-4x+2=0$
  \question $x^{2}+4x-1=0$
  \question $2x^{2}+3x-6=0$
  \question $2x^{2}+7x-6=0$
  \question $3x^{2}+9x-5=0$
  \question $-2x^{2}+4x+9=0$
  \question $-3x^{2}+6x+11=0$
  \question $-5x^{2}-7x+9=0$
  \question $-x^{2}-4x+9=0$
%----------------------------------------------------------------------------------------------------------------------------------------
\fullwidth{\noindent{\large \textbf{V. Naturaleza de las ecuaciones.}}}
\fullwidth{Determine la naturaleza de las soluciones de las siguientes ecuaciones de segundo grado mediante análisis del discriminante.}
  \question $2x^{2}-2x+3=0$
  \question $3x^{2}-4x-2=0$
  \question $x^{2}+7x-3=0$
  \question $x^{2}-3x+2=0$
  \question $3x^{2}+3x-1=0$
  \question $5x^{2}+4x-3=0$
  \question $x^{2}+x+5=0$
  \question $16x^{2}-8x+1=0$
  \question $3x^{2}+4x+1=0$
  \question $5x^{2}+4x-3=0$
  \question $4x^{2}-3x+2=0$
  \question $8x^{2}+2x-3=0$
%----------------------------------------------------------------------------------------------------------------------------------------
\fullwidth{\noindent{\large \textbf{VI. Propiedades de las soluciones.}}}
\fullwidth{Calcule la \textbf{suma} y el \textbf{producto} de las soluciones de las siguientes ecuaciones de segundo grado.}
  \question $25x^{2}-20x+1=0$
  \question $3x^{2}-2x+7=0$
  \question $x^{2}+11x-13=0$
  \question $5x^{2}-6x-14=0$
  \question $x^{2}+6x+5=0$
  \question $x^{2}-6x+5=0$
  \question $3x^{2}+x+2=0$
  \question $2x^{2}+x-1=0$
  \question $3x^{2}-5x-2=0$
  \question $3x^{2}+5x+2=0$
 













\end{questions}
\end{multicols}
%----------------------------------------------------------------------------------------------------------------------------------------
\end{document}
%----------------------------------------------------------------------------------------------------------------------------------------
%----------------------------------------------------------------------------------------------------------------------------------------
%%% Local Variables:
%%% mode: latex
%%% TeX-master: t
%%% End:

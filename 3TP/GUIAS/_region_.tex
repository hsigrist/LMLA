\message{ !name(Guia-Formativa-1-3TP-Numeros-Complejos.tex)}\documentclass[twocolumns,12pt,addpoints,x11names]{exam}
%____________________________________________________________________________________
\usepackage{pgf,tikz}
\usepackage{multicol}
\usepackage{tkz-euclide}
\usepackage{fourier}
\usepackage{fontspec}
\usepackage{graphicx}
\usepackage{amssymb,amsmath}
\usepackage{polyglossia}
\setdefaultlanguage{spanish}
\usetikzlibrary{arrows}
\usepackage{siunitx}
\usepackage{xcolor}
\usepackage{multicol}
\usepackage{hyperref}
\usetkzobj{all}
\usepackage[left=1.5cm,right=1.5cm,top=2.5cm,bottom=1.5cm,paperheight=33cm]{geometry}
%____________________________________________________________________________________
\setromanfont[Mapping={tex-text}]{Linux Biolinum O}
\setsansfont[Mapping={tex-text}]{Cantarell}
\setmonofont[Mapping={tex-text},Scale=0.8]{Pragmata Pro}
%____________________________________________________________________________________
\graphicspath{{/home/hsigrist/Dropbox/images/}
\everymath{\displaystyle}
\def\NN{\mathbb{N}}
\def\RR{\mathbb{R}}
\def\QQ{\mathbb{Q}}
\def\ZZ{\mathbb{Z}}
\def\II{\mathbb{I}}
\newcommand{\framedhref}[2]{\href{#1}{\fcolorbox{SlateGray1}{SlateGray1}{#2}}}
%____________________________________________________________________________________
\everymath{\displaystyle}
\extraheadheight{1in}
\extrafootheight{.75in}
%____________________________________________________________________________________
\pagestyle{headandfoot}
\firstpageheader{\includegraphics[scale=.56]{logolmlabw}}
{\Large\textbf{Guía Formativa 1}\\
  Números Complejos\\
  Tercero Medio TP\\
  marzo 2018}
{}
\runningheader{Guía 1 - Números Complejos / Tercero Medio TP}{}{Dpto. Matemática - LMLA $\cdot$ 2018, pág. \thepage}
\footer{}{}{}
%____________________________________________________________________________________
\def\mytitle{sxsx}
\def\mykeywords{xs}
\def\mysubject{xs}
\def\myauthor{Prof. Hans Sigrist}
\definecolor{links}{HTML}{000000}
%____________________________________________________________________________________
\hypersetup{%
  colorlinks,%
  linkcolor=,%
  citecolor=black,%
  urlcolor=links,%
  pdftitle={\mytitle},%
  pdfauthor={\myauthor},%
  pdfsubject={\mysubject},%
  pdfkeywords={\mykeywords}%
}
%____________________________________________________________________________________
% \printanswers
%____________________________________________________________________________________
\newcommand*\Matching[1]{%
  \ifprintanswers%
  \textbf{#1}%
  \else%
  \rule{2.1in}{0.5pt}%
  \fi%
}
%____________________________________________________________________________________
\newlength\matchlena
\newlength\matchlenb
\settowidth\matchlena{\rule{2.1in}{0pt}}
\newcommand\MatchQuestion[2]{%
  \setlength\matchlenb{\linewidth}%
  \addtolength\matchlenb{-\matchlena}%
  \parbox[t]{\matchlena}{\Matching{#1}}\enspace\parbox[t]{\matchlenb}{#2}%
}
%____________________________________________________________________________________
\begin{document}

\message{ !name(Guia-Formativa-1-3TP-Numeros-Complejos.tex) !offset(-3) }

%____________________________________________________________________________________
\addpoints
\renewcommand{\solutiontitle}{\noindent\textbf{Solución:}\enspace}
\pointname{ puntos}
\pointpoints{ punto}{ puntos}
\bracketedpoints
\boxedpoints
\CorrectChoiceEmphasis{\color{red}\bfseries}
\renewcommand{\choicelabel}{\thechoice)}
\chqword{Pregunta}
\chpword{Puntos}
\chbpword{Bonus}
\chsword{Total}\hqword{Pregunta}
\hpword{Puntos}
\cellwidth{3mm}
\setlength\answerskip{2ex}
\setlength\answerlinelength{2in}
%____________________________________________________________________________________
\fullwidth{\noindent\large\textbf{Nombre:}\enspace\makebox[5.2in]{\hrulefill} \textbf{Curso:}\enspace\makebox[0.5in]{\hrulefill}}
\fullwidth{%
  \begin{description}
  \item[Objetivos:] Comprender qué es un número imaginario, cómo se operan y calcular potencias de $i$.
  \end{description}
}
%____________________________________________________________________________________
\fullwidth{\noindent{\large \textbf{I. Escriba en términos de $i$}}}
\begin{questions}

\question $\sqrt{-9}$
  \answerline
  \question $\sqrt{-64}$
  \answerline
  \question $\sqrt{-\frac{1}{4}}$
  \answerline
  \question $\sqrt{-5}$
  \answerline
  \question $\sqrt{-8}$
  \answerline
 


\fullwidth{\noindent{\large \textbf{II. Calcula el valor de las siguientes potencias de $i$, reduce al máximo tu resultado.}}}

\question $i^7-7i^7=$
  \answerline

\question $2i^9+12i^{12}=$
  \answerline

\question $3i^3+12i^{12}-21i^{21}+11i^{11}=$
  \answerline

\question $(2i^8+3i^9)-(12i^7-12i^9)=$
  \answerline

\question $i^{33}(i^{267}+i^{342}-i^{455}+i^{222})=$
  \answerline

\newpage


\twocolumn
\fullwidth{\noindent{\large \textbf{III. Módulo y conjugado de complejos.}}}

Dados $z_1=1-i$ y $z_2=2+i$, obtenga:


\question $\overline{z_1}$
   \begin{solutionbox}{1in}
    
  \end{solutionbox}

\question $\overline{z_2}$
   \begin{solutionbox}{1in}
    
  \end{solutionbox}

\question $\overline{\overline{z_1}}$
   \begin{solutionbox}{1in}
    
  \end{solutionbox}

\question $\overline{\overline{z_2}}$
   \begin{solutionbox}{1in}
    
  \end{solutionbox}

\question $\overline{(z_1+z_2)}$
   \begin{solutionbox}{1in}
    
  \end{solutionbox}

\vspace{3cm}
  
\question $\overline{(z_1-z_2)}$
   \begin{solutionbox}{1in}
    
  \end{solutionbox}

\question $\overline{z_1}-\overline{z_2}$
   \begin{solutionbox}{1in}
    
  \end{solutionbox}

\question $\overline{z_1}+\overline{z_2}$
   \begin{solutionbox}{1in}
    
  \end{solutionbox}

\question $|z_1|$
   \begin{solutionbox}{1in}
    
  \end{solutionbox}

\question $|\overline{z_2}|$
   \begin{solutionbox}{1in}
    
  \end{solutionbox}

\question $|z_1|^{2}$
  \begin{solutionbox}{1in}
    
  \end{solutionbox}
  
\end{questions}
\end{document}
\message{ !name(Guia-Formativa-1-3TP-Numeros-Complejos.tex) !offset(-213) }

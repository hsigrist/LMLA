\documentclass[12pt,spanish,x11names]{beamer}
%-------------------------------------------------------------------------------------------------------
% \usepackage{pgfpages}
% \setbeameroption{hide notes}
% \setbeameroption{show notes}
% \setbeameroption{show notes on second screen=right}
\usetheme{Hytex}
\setbeamertemplate{navigation symbols}{}
\usecolortheme[RGB={7,29,66}]{structure}
\usepackage{tcolorbox}
\usepackage{fourier}
\usepackage{float}
\usepackage{fontspec}
\usepackage{graphicx}
\usepackage{amssymb,amsmath}
\usepackage{polyglossia}
\setdefaultlanguage{spanish}
\usepackage[style=spanish]{csquotes}
\usepackage{pstricks-add}
\usepackage{tkz-euclide}
\usetkzobj{all}
\usepackage{pgf,tikz}
\usetikzlibrary{mindmap,trees,arrows}
\usepackage{siunitx}
\usepackage{xcolor}
\usepackage{booktabs}
\usepackage{marvosym}
\setbeamertemplate{caption}[numbered]
\usepackage{hyperref}
%-------------------------------------------------------------------------------------------------------
\def\talkclass{Presentación}
\def\talkcar{3TP}
\def\talkdate{\today}
\def\talkversion{}
\def\talktitle{Ecuación de Segundo Grado}
\def\talksubtitle{Elementos, resolución, usos}
\def\talkkeywords{ecuación de segundo grado}
\def\talksubject{ecuación de segundo grado}
\def\talkblog{https://hsigrist.github.io}
\def\talkpubpdf{https://github.com/hsigrist/LMLA/blob/b3ee3cf0a7fda2d4ee27a545bcb482b6cf780d33/3TP/TALKS/ECUACION-SEGUNDO-GRADO-3TP.pdf}
\def\talkcopyright{\myauthor}
\def\talkaffiliation{Liceo Mixto Los Andes}
\def\talkauthor{Hans Sigrist}
\def\talkgrade{Lic. \& Mag. Matemática}
\def\talkemail{hsigrist@liceomixto.cl}
\definecolor{links}{HTML}{000000}
\def\NN{\mathbb{N}}
\def\RR{\mathbb{R}}
\def\ZZ{\mathbb{Z}}
\def\QQ{\mathbb{Q}}
\def\II{\mathbb{I}}
\definecolor{bluu}{RGB}{7,29,66}
\newcommand{\framedhref}[2]{\href{#1}{\fcolorbox{bluu}{bluu}{\textcolor{white}{#2}}}}
\newtheorem{teorema}{Teorema}[section]
\newtheorem{lema}[teorema]{Lema}
\newtheorem{proposicion}[teorema]{Proposición}
\newtheorem{corolario}[teorema]{Corolario}
\newtheorem{definicion}[teorema]{Definición}
\newtheorem{ejemplo}[teorema]{Ejemplo}
\newtheorem{nota}[teorema]{Nota}
%-------------------------------------------------------------------------------------------------------
\hypersetup{pdfpagemode=FullScreen,colorlinks,linkcolor=,citecolor=black,urlcolor=links,pdftitle={pdftitle},pdfauthor={\talkauthor},pdfsubject={\talksubject},pdfkeywords={\talkkeywords}}
%-------------------------------------------------------------------------------------------------------
\setmainfont[Mapping={tex-text},Numbers={OldStyle},Ligatures=TeX]{Linux Biolinum
    O}
\setsansfont[Mapping={tex-text},Numbers={OldStyle},Ligatures=TeX]{Linux Libertine
    O}
\setmonofont[Mapping={tex-text},Numbers={OldStyle},Ligatures=Rare,Scale=0.8]{Pragmata
    Pro Mono}
%-------------------------------------------------------------------------------------------------------
\graphicspath{{/home/hsigrist/Dropbox/images/}}
\everymath{\displaystyle}
\AtBeginSection[]{\begin{frame}<beamer>\frametitle{Agenda}\tableofcontents[sectionstyle=show/hide,subsectionstyle=hide/show/hide,currentsection]\end{frame}\addtocounter{framenumber}{-1}}
%-------------------------------------------------------------------------------------------------------
\title{\talktitle}
\subtitle{\talksubtitle}
\author{\talkauthor}
\institute{\talkaffiliation}
\date{\footnotesize{\emph{\href{\talkblog}{\talkemail}}}}
%-------------------------------------------------------------------------------------------------------
\begin{document}
\begin{frame}
\titlepage
\end{frame}
%-------------------------------------------------------------------------------------------------------
\section{Ecuación de segundo grado}
\begin{frame}
  \frametitle{Variable cuadrática}
  \begin{exampleblock}{Problema}
    Un distribuidor compró cierta cantidad de objetos por $\$280$. Si hubiera
    comprado cuatro menos, cada objeto habría costado $\$8$ más. ¿Cuántos
    objetos compró?
  \end{exampleblock}
\end{frame}
%-------------------------------------------------------------------------------------------------------
\section{Elementos}
\begin{frame}
  \frametitle{Coeficientes cuadráticos}
  \begin{block}{Ecuación de segundo grado}
    \begin{align*}
      \label{eq:1}
      ax^2+bx+c&=0
    \end{align*}
    \begin{description}
    \item[$a$] coeficiente cuadrático (grado $2$)
    \item[$b$] coeficiente lineal (grado $1$)
    \item[$c$] coeficiente constante (grado $0$)
    \end{description}
  \end{block}
\end{frame}
%-------------------------------------------------------------------------------------------------------
\section{Tipos de ecuaciones de segundo grado}
\begin{frame}
  \frametitle{Según sus coeficientes}
  \begin{exampleblock}{Se expresan como:}
    \begin{align*}
      x^2-4x-140&=0\hspace{1cm}\text{\emph{ecuación seg. grado completa}}\\
      5x^2-20&=0\hspace{1cm}\text{\emph{ecuación seg. pura}}\\
      16x^2-8x&=0\hspace{1cm}\text{\emph{ecuación seg. grado factorizable}}
    \end{align*}
  \end{exampleblock}
\end{frame}












\begin{frame}[c]\frametitle{Apéndice}
\centering\decofourleft\quad\decofourright

\textbf{\emph {Saludos a todas/os, ¡Muchas gracias!}}

Una copia del presente trabajo, se encuentra en el enlace \framedhref{\talkpubpdf}{\talktitle}.

\end{frame}
%-------------------------------------------------------------------------------------------------------
\end{document}
%-------------------------------------------------------------------------------------------------------
% !TEX program = xelatex
\documentclass[12pt,spanish,x11names]{beamer}
%-------------------------------------------------------------------------------------------------------
% \usepackage{pgfpages}
% \setbeameroption{hide notes}
% \setbeameroption{show notes}
% \setbeameroption{show notes on second screen=right}
\usetheme{Hytex}
\setbeamertemplate{navigation symbols}{}
\usecolortheme[RGB={7,29,66}]{structure}
\usepackage{array}
\usepackage{tcolorbox}
\usepackage{fourier}
\usepackage{float}
\usepackage{fontspec}
\usepackage{graphicx}
\usepackage{amssymb,amsmath}
\usepackage{polyglossia}
\setdefaultlanguage{spanish}
\usepackage[style=spanish]{csquotes}
\usepackage{pstricks-add}
\usepackage{tkz-euclide}
\usetkzobj{all}
\usepackage{pgf,tikz}
\usetikzlibrary{mindmap,trees,arrows}
\usepackage{siunitx}
\usepackage{xcolor}
\usepackage{booktabs}
\usepackage{marvosym}
\setbeamertemplate{caption}[numbered]
\usepackage{hyperref}
%-------------------------------------------------------------------------------------------------------
\def\talkclass{Presentación}
\def\talkcar{3TP}
\def\talkdate{\today}
\def\talkversion{}
\def\talktitle{La Función Cuadrática}
\def\talksubtitle{Elementos de la parábola, modelos cuadráticos}
\def\talkkeywords{parábola, concavidad, discriminante}
\def\talksubject{función cuadrática}
\def\talkblog{https://hsigrist.github.io}
\def\talkpubpdf{https://github.com/hsigrist/LMLA/blob/master/3TP/TALKS/FUNCION-CUADRATICA.pdf}
\def\talkcopyright{\myauthor}
\def\talkaffiliation{Liceo Mixto Los Andes}
\def\talkauthor{Hans Sigrist}
\def\talkgrade{Lic. \& Mag. Matemática}
\def\talkemail{hsigrist@liceomixto.cl}
\definecolor{links}{HTML}{000000}
\def\NN{\mathbb{N}}
\def\RR{\mathbb{R}}
\def\CC{\mathbb{C}}
\def\ZZ{\mathbb{Z}}
\def\QQ{\mathbb{Q}}
\def\II{\mathbb{I}}
\definecolor{bluu}{RGB}{7,29,66}
\newcommand{\framedhref}[2]{\href{#1}{\fcolorbox{bluu}{bluu}{\textcolor{white}{#2}}}}
\newtheorem{teorema}{Teorema}[section]
\newtheorem{lema}[teorema]{Lema}
\newtheorem{proposicion}[teorema]{Proposición}
\newtheorem{corolario}[teorema]{Corolario}
\newtheorem{definicion}[teorema]{Definición}
\newtheorem{ejemplo}[teorema]{Ejemplo}
\newtheorem{nota}[teorema]{Nota}
%-------------------------------------------------------------------------------------------------------
\newcolumntype{+}{>{\global\let\currentrowstyle\relax}}
\newcolumntype{^}{>{\currentrowstyle}}
\newcommand{\rowstyle}[1]{\gdef\currentrowstyle{#1}%
#1\ignorespaces
}
%-------------------------------------------------------------------------------------------------------
\hypersetup{pdfpagemode=FullScreen,colorlinks,linkcolor=,citecolor=black,urlcolor=links,pdftitle={pdftitle},pdfauthor={\talkauthor},pdfsubject={\talksubject},pdfkeywords={\talkkeywords}}
%-------------------------------------------------------------------------------------------------------
\setmainfont[Mapping={tex-text},Numbers={OldStyle},Ligatures=TeX]{Linux Biolinum
    O}
\setsansfont[Mapping={tex-text},Numbers={OldStyle},Ligatures=TeX]{Linux Libertine
    O}
\setmonofont[Mapping={tex-text},Numbers={OldStyle},Scale=0.8]{Pragmata
    Pro Mono}
%-------------------------------------------------------------------------------------------------------
\graphicspath{{/home/hsigrist/Dropbox/images/}}
\everymath{\displaystyle}
\AtBeginSection[]{\begin{frame}<beamer>\frametitle{Agenda}\tableofcontents[sectionstyle=show/hide,subsectionstyle=hide/show/hide,currentsection]\end{frame}\addtocounter{framenumber}{-1}}
%-------------------------------------------------------------------------------------------------------
\title{\talktitle}
\subtitle{\talksubtitle}
\author{\talkauthor}
\institute{\talkaffiliation}
\date{\footnotesize{\emph{\href{\talkblog}{\talkemail}}}}
%-------------------------------------------------------------------------------------------------------
\begin{document}
\begin{frame}
\titlepage
\end{frame}
%-------------------------------------------------------------------------------------------------------
\section{Discriminante}
\begin{frame}
  \frametitle{Naturaleza de las soluciones}
  \begin{block}{El símbolo $\Delta$}
    Anteriormente vimos que una ecuación de seg. grado general se podía resolver
    por medio de la \textbf{fórmula general}, dada por
    \begin{align*}
      x&=\frac{-b\pm\sqrt{b^2-4ac}}{2a}
    \end{align*}
    en esta expresión, la cantidad subradical de la raíz $b^2-4ac$ se denomina
    \textbf{discriminante} y se simboliza por $\Delta$.
  \end{block}
\end{frame}
%-------------------------------------------------------------------------------------------------------
\begin{frame}
  \frametitle{Naturaleza de las soluciones}
  \begin{block}{Propiedades}
    \begin{align*}
    \Delta>0&\Rightarrow 2\text{ sol.},\RR,\neq\\
    \Delta=0&\Rightarrow 2\text{ sol.},\RR,=\\
    \Delta<0&\Rightarrow 2\text{ sol.},\CC,\text{ conjugadas}
    \end{align*}
  \end{block}
\end{frame}
%-------------------------------------------------------------------------------------------------------
\section{La función cuadrática}
\begin{frame}
  \frametitle{Es una función epiyectiva}
  \begin{block}{Definición}
    Una función cuadrática es de la forma $f(x)=ax^2+bx+c$, donde $a,b,c\in\RR$
    y $a\neq0$.
  \end{block}
\end{frame}
%-------------------------------------------------------------------------------------------------------
\begin{frame}
  \frametitle{Caso $a=1$, $b=0$ y $c=0$}
  \pause
  En este caso tenemos la función $f(x)=x^2$. Con esta función construyamos una
  tabla de valores de la forma $(x,f(x))$:

  \centering
  \begin{tabular}[t]{|c|c|}
    \hline
    $x$ & $f(x)$\\
    \hline
    $-4$ & \\      
    \hline
    $-3$ & \\      
    \hline
    $-2$ & \\      
    \hline
    $-1$ & \\      
    \hline
    $0$ & \\      
    \hline
    $1$ & \\      
    \hline
    $2$ & \\      
    \hline
    $3$ & \\      
    \hline
    $4$ & \\
    \hline
  \end{tabular}
\end{frame}
%-------------------------------------------------------------------------------------------------------
\begin{frame}
  \frametitle{La parábola}
  Con los pares ordenados anteriores diseñemos la parábola:
    \begin{minipage}[t]{.6\linewidth}
     \begin{figure}[h]
      \centering
      \begin{tikzpicture}[scale=.5]
        \tkzInit[xmax=5,xmin=-5,ymax=10,ymin=0]
        \tkzAxeXY
        \tkzGrid
        \tkzDefPoint(-3,9){B}
        \tkzDefPoint(-2,4){C}
        \tkzDefPoint(-1,1){D}
        \tkzDefPoint(0,0){E}
        \tkzDefPoint(1,1){F}
        \tkzDefPoint(2,4){G}
        \tkzDefPoint(3,9){H}
        \tkzDrawPoint(B)
        \tkzDrawPoint(C)
        \tkzDrawPoint(D)
        \tkzDrawPoint(E)
        \tkzDrawPoint(F)
        \tkzDrawPoint(G)
        \tkzDrawPoint(H)
      \end{tikzpicture}
    \end{figure}
  \end{minipage}
  \begin{minipage}[t]{.3\linewidth}
  \begin{tabular}[t]{|c|c|}
    \hline
    $x$ & $f(x)$\\
    \hline
    $-3$ & $9$\\      
    \hline
    $-2$ & $4$\\      
    \hline
    $-1$ & $1$\\      
    \hline
    $0$ & $0$\\      
    \hline
    $1$ & $1$\\      
    \hline
    $2$ & $4$\\      
    \hline
    $3$ & $9$\\      
    \hline
  \end{tabular}
  \end{minipage}

\end{frame}



























%-------------------------------------------------------------------------------------------------------
\begin{frame}[c]\frametitle{Apéndice}
\centering\decofourleft\quad\decofourright

\textbf{\emph {Carpe diem!}}

Una copia del presente trabajo, se encuentra en el enlace \framedhref{\talkpubpdf}{\talktitle}.

\end{frame}
%-------------------------------------------------------------------------------------------------------
\end{document}
%-------------------------------------------------------------------------------------------------------
% !TEX program = xelatex
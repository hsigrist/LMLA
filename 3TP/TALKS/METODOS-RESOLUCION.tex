\documentclass[12pt,spanish,x11names]{beamer}
%-------------------------------------------------------------------------------------------------------
% \usepackage{pgfpages}
% \setbeameroption{hide notes}
% \setbeameroption{show notes}
% \setbeameroption{show notes on second screen=right}
\usetheme{Hytex}
\setbeamertemplate{navigation symbols}{}
\usecolortheme[RGB={7,29,66}]{structure}
\usepackage{array}
\usepackage{tcolorbox}
\usepackage{fourier}
\usepackage{float}
\usepackage{fontspec}
\usepackage{graphicx}
\usepackage{amssymb,amsmath}
\usepackage{polyglossia}
\setdefaultlanguage{spanish}
\usepackage[style=spanish]{csquotes}
\usepackage{pstricks-add}
\usepackage{tkz-euclide}
\usetkzobj{all}
\usepackage{pgf,tikz}
\usetikzlibrary{mindmap,trees,arrows}
\usepackage{siunitx}
\usepackage{xcolor}
\usepackage{booktabs}
\usepackage{marvosym}
\setbeamertemplate{caption}[numbered]
\usepackage{hyperref}
%-------------------------------------------------------------------------------------------------------
\def\talkclass{Presentación}
\def\talkcar{3TP}
\def\talkdate{\today}
\def\talkversion{}
\def\talktitle{Métodos de Resolución}
\def\talksubtitle{Resolución de ec. seg. grado según su naturaleza}
\def\talkkeywords{métodos resolución, estrategias}
\def\talksubject{ecuación de segundo grado}
\def\talkblog{https://hsigrist.github.io}
\def\talkpubpdf{https://github.com/hsigrist/LMLA/blob/master/3TP/TALKS/ECUACION-SEGUNDO-GRADO-3TP.pdf}
\def\talkcopyright{\myauthor}
\def\talkaffiliation{Liceo Mixto Los Andes}
\def\talkauthor{Hans Sigrist}
\def\talkgrade{Lic. \& Mag. Matemática}
\def\talkemail{hsigrist@liceomixto.cl}
\definecolor{links}{HTML}{000000}
\def\NN{\mathbb{N}}
\def\RR{\mathbb{R}}
\def\CC{\mathbb{C}}
\def\ZZ{\mathbb{Z}}
\def\QQ{\mathbb{Q}}
\def\II{\mathbb{I}}
\definecolor{bluu}{RGB}{7,29,66}
\newcommand{\framedhref}[2]{\href{#1}{\fcolorbox{bluu}{bluu}{\textcolor{white}{#2}}}}
\newtheorem{teorema}{Teorema}[section]
\newtheorem{lema}[teorema]{Lema}
\newtheorem{proposicion}[teorema]{Proposición}
\newtheorem{corolario}[teorema]{Corolario}
\newtheorem{definicion}[teorema]{Definición}
\newtheorem{ejemplo}[teorema]{Ejemplo}
\newtheorem{nota}[teorema]{Nota}
%-------------------------------------------------------------------------------------------------------
\newcolumntype{+}{>{\global\let\currentrowstyle\relax}}
\newcolumntype{^}{>{\currentrowstyle}}
\newcommand{\rowstyle}[1]{\gdef\currentrowstyle{#1}%
#1\ignorespaces
}
%-------------------------------------------------------------------------------------------------------
\hypersetup{pdfpagemode=FullScreen,colorlinks,linkcolor=,citecolor=black,urlcolor=links,pdftitle={pdftitle},pdfauthor={\talkauthor},pdfsubject={\talksubject},pdfkeywords={\talkkeywords}}
%-------------------------------------------------------------------------------------------------------
\setmainfont[Mapping={tex-text},Numbers={OldStyle},Ligatures=TeX]{Linux Biolinum
    O}
\setsansfont[Mapping={tex-text},Numbers={OldStyle},Ligatures=TeX]{Linux Libertine
    O}
\setmonofont[Mapping={tex-text},Numbers={OldStyle},Scale=0.8]{Pragmata
    Pro Mono}
%-------------------------------------------------------------------------------------------------------
\graphicspath{{/home/hsigrist/Dropbox/images/}}
\everymath{\displaystyle}
\AtBeginSection[]{\begin{frame}<beamer>\frametitle{Agenda}\tableofcontents[sectionstyle=show/hide,subsectionstyle=hide/show/hide,currentsection]\end{frame}\addtocounter{framenumber}{-1}}
%-------------------------------------------------------------------------------------------------------
\title{\talktitle}
\subtitle{\talksubtitle}
\author{\talkauthor}
\institute{\talkaffiliation}
\date{\footnotesize{\emph{\href{\talkblog}{\talkemail}}}}
%-------------------------------------------------------------------------------------------------------
\begin{document}
\begin{frame}
\titlepage
\end{frame}
%-------------------------------------------------------------------------------------------------------
\section{Tipo I}
\begin{frame}
  \frametitle{Ecuaciones seg. grado puras}
  \begin{block}{Propiedad}
    \begin{align*}
      ax^2+c&=0\\
      a,c\in\RR\wedge a&\neq0
    \end{align*}
    Si la ecuación tiene solución en los números reales, las raíces de las
    ecuaciones son una la inversa aditiva de la otra, es decir, si una es $a$,
    entonces la otra será $-a$. Si la ecuación no tiene solución en los números
    reales, estas son complejas conjugadas.
  \end{block}
\end{frame}
%-------------------------------------------------------------------------------------------------------
\begin{frame}
  \frametitle{Ecuaciones seg. grado puras}
  \begin{exampleblock}{Problemas}
    Resuelve las siguientes ecuaciones de seg. grado puras y verifica la
    propiedad anterior.
    \begin{align}
      (x+40)(x-40)&=900\\
      \frac{6x^2-11}{114+x^2}&=1\\
      (14x+5)(1-x)&=x(9-x)\\
      2\sqrt{1+0.5x^2}&=\sqrt{166}\\
      \frac{1}{4z^2}-\frac{1}{9z^2}&=\frac{1}{24}
    \end{align}
  \end{exampleblock}
\end{frame}
%-------------------------------------------------------------------------------------------------------
\section{Tipo II}
\begin{frame}
  \frametitle{Ecuaciones seg. grado factorizables}
  \begin{block}{Propiedad}
    \begin{align*}
      ax^2+bx&=0\\
      a,b\in\RR\wedge a&\neq0
    \end{align*}
    En este caso siempre hay dos soluciones reales, donde una de ellas es
    siempre cero y la otra es un número real cualquiera.
  \end{block}
\end{frame}
%-------------------------------------------------------------------------------------------------------
\begin{frame}
  \frametitle{Ecuaciones seg. grado factorizables}
  \begin{exampleblock}{Problemas}
    Resuelve las siguientes ecuaciones de seg. grado factorizables y verifica la
    propiedad anterior.
    \begin{align}
      5x^2-41x&=0\\
      \frac{19x^2+12}{(3x+2)(2x-3)}&=-2\\
      \frac{x^2+4}{(x-3)(x-5)}&=\frac{4}{15}\\
      (7-11x)(11-7x)-(1-x^2)&=4(19-4x)\\
      4+\left(6y-\frac{3}{4}\right)&=\frac{73}{16}
    \end{align}
  \end{exampleblock}
\end{frame}
%-------------------------------------------------------------------------------------------------------
\section{Tipo III}
\begin{frame}
  \frametitle{Ecuaciones seg. grado trinomio factorizable}
  \begin{block}{Propiedad}
    \begin{align*}
      ax^2+bx+c&=0\\
      a,b,c\in\RR\wedge a&\neq0
    \end{align*}
    En este caso pueden existir dos soluciones reales y distintas o dos soluciones reales e iguales.
  \end{block}
\end{frame}
%-------------------------------------------------------------------------------------------------------
\begin{frame}
  \frametitle{Ecuaciones seg. grado trinomio factorizable}
  \begin{exampleblock}{Problemas}
    Resuelve las siguientes ecuaciones de seg. grado trinomio factorizable y verifica la
    propiedad anterior.
    \begin{align}
      \frac{2x+11}{x}+\frac{x-5}{3}&=5\\
      (x-\sqrt{3x})(8x+\sqrt{3x})&=18\\
      4x^2-4x+1&=0\\
      1&=\frac{x+1}{\sqrt{7+x}}\\
      \sqrt{2x+\frac{1}{8x}}&=1
    \end{align}
  \end{exampleblock}
\end{frame}
%-------------------------------------------------------------------------------------------------------
\section{Tipo IV}
\begin{frame}
  \frametitle{Ecuaciones seg. grado forma general}
  \begin{block}{Propiedad}
    \begin{align*}
      ax^2+bx+c&=0\\
      a,b,c\in\RR\wedge a&\neq0
    \end{align*}
    En este caso su solución viene dada por la expresión
    \begin{align*}
      x&=\frac{-b\pm\sqrt{b^2-4ac}}{2a}
    \end{align*}
    conocida como la \textbf{formula general} para resolver \textbf{cualquier}
    ecuación de segundo grado.
  \end{block}
\end{frame}
%-------------------------------------------------------------------------------------------------------
\begin{frame}
  \frametitle{Ecuaciones seg. grado general}
  \begin{exampleblock}{Problemas}
    Resuelve las siguientes ecuaciones de seg. grado trinomio y establezca la
    naturaleza de las soluciones ($\in\RR\vee\CC$).
    \begin{align}
      \sqrt{5x+6}-\sqrt{x+7}&=1\\
      2\sqrt{2x+1}&=\sqrt{x+5}+\sqrt{4x-7}\\
      (2+y)^2&=100-y^2\\
      \frac{x+5}{x}+\frac{2x-3}{x}&=\frac{x-3}{x-2}\\
      \frac{9}{4-x}&=\frac{x+4}{x+2}
    \end{align}
  \end{exampleblock}
\end{frame}
%-------------------------------------------------------------------------------------------------------






%-------------------------------------------------------------------------------------------------------
\begin{frame}[c]\frametitle{Apéndice}
\centering\decofourleft\quad\decofourright

\textbf{\emph {Carpe diem!}}

Una copia del presente trabajo, se encuentra en el enlace \framedhref{\talkpubpdf}{\talktitle}.

\end{frame}
%-------------------------------------------------------------------------------------------------------
\end{document}
%-------------------------------------------------------------------------------------------------------
% !TEX program = xelatex
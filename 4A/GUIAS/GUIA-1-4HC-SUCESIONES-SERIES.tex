\documentclass[12pt,addpoints,x11names]{exam}
\usepackage{pgf,tikz}
\usepackage{multicol}
\usepackage{tkz-euclide}
\usepackage{fourier}
\usepackage{fontspec}
\usepackage{graphicx}
\usepackage{scalerel,amssymb,amsmath}
\usepackage{polyglossia}
\setdefaultlanguage{spanish}
\usetikzlibrary{arrows}
\usepackage{siunitx}
\usepackage{xcolor}
\usepackage{multicol}
\usepackage{hyperref}
\usetkzobj{all}
\usepackage[left=1.5cm,right=1.5cm,top=2.5cm,bottom=1.5cm,paperheight=33cm]{geometry}
\setromanfont[Mapping={tex-text}]{Linux Biolinum O}
\setsansfont[Mapping={tex-text}]{Cantarell}
\setmonofont[Mapping={tex-text},Scale=0.8]{Pragmata Pro}
\graphicspath{{/home/hsigrist/Dropbox/images/}}
\setlength\columnsep{30pt}
\everymath{\displaystyle}
\def\NN{\mathbb{N}}
\def\RR{\mathbb{R}}
\def\QQ{\mathbb{Q}}
\def\ZZ{\mathbb{Z}}
\def\II{\mathbb{I}}
\newcommand{\framedhref}[2]{\href{#1}{\fcolorbox{SlateGray1}{SlateGray1}{#2}}}
\everymath{\displaystyle}
\extraheadheight{1in}
\extrafootheight{.75in}
\pagestyle{headandfoot}
\firstpageheader{\includegraphics[scale=.56]{logolmlabw}}
{\Large\textbf{Guía 1}\\
  Sucesiones Numéricas\\
  Cuarto Medio HC\\
  marzo 2018}
{}
\runningheader{Guía 1 - Sucesiones y Series / Cuarto Medio HC}{}{Dpto.
  Matemática - LMLA $\cdot$ 2018, pág. \thepage}
\footer{}{}{}
\def\mytitle{Guía 1 4HC Sucesiones y Series}
\def\mykeywords{sucesiones, recurrencia, PA, PG}
\def\mysubject{series}
\def\myauthor{Prof. Hans Sigrist}
\definecolor{links}{HTML}{000000}
\hypersetup{%
  colorlinks,%
  linkcolor=,%
  citecolor=black,%
  urlcolor=links,%
  pdftitle={\mytitle},%
  pdfauthor={\myauthor},%
  pdfsubject={\mysubject},%
  pdfkeywords={\mykeywords}%
}
% \printanswers
\newcommand*\Matching[1]{%
  \ifprintanswers%
  \textbf{#1}%
  \else%
  \rule{2.1in}{0.5pt}%
  \fi%
}
\newlength\matchlena
\newlength\matchlenb
\settowidth\matchlena{\rule{2.1in}{0pt}}
\newcommand\MatchQuestion[2]{%
  \setlength\matchlenb{\linewidth}%
  \addtolength\matchlenb{-\matchlena}%
  \parbox[t]{\matchlena}{\Matching{#1}}\enspace\parbox[t]{\matchlenb}{#2}%
}
\begin{document}
\addpoints
\renewcommand{\solutiontitle}{\noindent\textbf{Solución:}\enspace}
\pointname{ puntos}
\pointpoints{ punto}{ puntos}
\bracketedpoints
\boxedpoints
\CorrectChoiceEmphasis{\color{red}\bfseries}
\renewcommand{\choicelabel}{\thechoice)}
\chqword{Pregunta}
\chpword{Puntos}
\chbpword{Bonus}
\chsword{Total}\hqword{Pregunta}
\hpword{Puntos}
\cellwidth{3mm}
\setlength\answerskip{2ex}
\setlength\answerlinelength{2in}
\fullwidth{\noindent\large\textbf{Nombre:}\enspace\makebox[5.2in]{\hrulefill}
  \textbf{Curso:}\enspace\makebox[0.5in]{\hrulefill}}
\fullwidth{%
  \begin{description}
  \item[Objetivos:] Analizan las transformaciones que producen diferentes tipos de
iteraciones y establecen relaciones cuantitativas y cualitativas entre los
objetos que se obtienen en una sucesión numérica.
  \end{description}
}
\fullwidth{\noindent{\large \textbf{I. Sucesiones numéricas}}}

\begin{questions}

\begin{multicols}{2}

\fullwidth{\textbf{Escriba los cuatro primeros términos de la sucesión que inicia con:}}

\question $4$ y suma $9$ a cada término siguiente.

\question $45$ y resta $6$ a cada término siguiente.

\question $2$ y multiplica por $3$ cada término siguiente.

\question $96$ y divide por $2$ cada término siguiente.


\fullwidth{\textbf{Para cada una de las siguientes sucesiones escriba su fórmula y
  encuentre los siguientes dos términos.}}

\question $8,\ 16,\ 24,\ 32,\ldots$

\question $2,\ 5,\ 8,\ 11,\ldots$

\question $36,\ 31,\ 26,\ 21,\ldots$

\question $243,\ 81,\ 27,\ 9,\ldots$


\fullwidth{\textbf{Las siguientes preguntas están relacionadas con el término
    general de una sucesión.}}

\question Una sucesión está definida por $u_n=3n-2$, hallar:
\begin{itemize}
  \item $u_1=$
  \item $u_5=$
  \item $u_{27}=$
\end{itemize}

\question Calcule los $5$ primeros términos de la sucesión definida por
$u_n=2n+5$.

\question Encuentre el término $27$ y el $41$ de la sucesión $5,\ 11,\ 17,\ldots$


\question Encuentre el término $13$ y el $109$ de la sucesión $71,\ 70,\ 69,\ldots$


\question Encuentre el término $17$ y el $54$ de la sucesión $10$,
$\scalerel{11}{\scriptstyle\frac{1}{2}}$, $13$,\ldots


\question Encuentre el término $20$ y el $13$ de la sucesión $-3,\ -2,\ -1,\ldots$


\question Encuentre el término $90$ y el $16$ de la sucesión $-4,\ 2.5,\ 9,\ldots$


\question Encuentre el término $37$ y el $89$ de la sucesión $-2.8,\ 0,\ 2.8,\ldots$



\fullwidth{\textbf{Encuentre el último término de las siguientes sucesiones:}}

\question $5,\ 7,\ 9,\ldots$ hasta completar $20$ términos.

\question $7,\ 3,\ -1,\ldots$ hasta completar $15$ términos.

\question $\scalerel{13}{\scriptstyle\frac{1}{2}}$, $9$,
$\scalerel{4}{\scriptstyle\frac{1}{2}}$,\ldots hasta completar $13$ términos.

\question $.6,\ 1.2,\ 1.8,\ldots$ hasta completar $12$ términos.



\fullwidth{\textbf{Encuentre el último término y la suma de las siguientes sucesiones:}}

\question $14,\ 64,\ 114$ hasta completar $20$ términos.

\question $1,\ 1.2,\ 1.4,\ldots$ hasta completar $12$ términos.

\question $9,\ 5,\ 1,\ldots$ hasta completar $100$ términos.

\question $\frac{1}{14},\ \frac{-1}{4},\ \frac{-3}{4}$ hasta completar $21$ términos.

\end{multicols}

\end{questions}
\end{document}

% Local Variables:
% TeX-engine: xetex
% End:

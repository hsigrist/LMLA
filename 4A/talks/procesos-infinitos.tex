\documentclass[12pt,spanish,x11names]{beamer}
%-------------------------------------------------------------------------------------------------------
% \usepackage{pgfpages}
% \setbeameroption{hide notes}
% \setbeameroption{show notes}
% \setbeameroption{show notes on second screen=right}
\usetheme{Hytex}
\setbeamertemplate{navigation symbols}{}
\usecolortheme[RGB={7,29,66}]{structure}
\usepackage{tcolorbox}
\usepackage{fourier}
\usepackage{float}
\usepackage{fontspec}
\usepackage{graphicx}
\usepackage{amssymb,amsmath}
\usepackage{polyglossia}
\setdefaultlanguage{spanish}
\usepackage[style=spanish]{csquotes}
\usepackage{pstricks-add}
\usepackage{tkz-euclide}
\usetkzobj{all}
\usepackage{pgf,tikz}
\usetikzlibrary{mindmap,trees,arrows}
\usepackage{siunitx}
\usepackage{xcolor}
\usepackage{booktabs}
\usepackage{marvosym}
\setbeamertemplate{caption}[numbered]
\usepackage{hyperref}
%-------------------------------------------------------------------------------------------------------
\def\talkclass{Presentación}
\def\talkcar{4HC}
\def\talkdate{\today}
\def\talkversion{}
\def\talktitle{Procesos Infinitos}
\def\talksubtitle{Sucesiones y series: sumas ad infinitum}
\def\talkkeywords{sucesiones, series, progresiones aritméticas, progresiones geométricas}
\def\talksubject{procesos infinitos}
\def\talkblog{https://hsigrist.github.io}
\def\talkpubpdf{https://www.dropbox.com/s/tuhjx03333fl5vu/todo-es-bullying.pdf?dl=0}
\def\talkcopyright{\myauthor}
\def\talkaffiliation{Liceo Mixto Los Andes}
\def\talkauthor{Hans Sigrist}
\def\talkgrade{Lic. \& Mag. Matemática}
\def\talkemail{hsigrist@liceomixto.cl}
\definecolor{links}{HTML}{000000}
\def\NN{\mathbb{N}}
\def\RR{\mathbb{R}}
\def\ZZ{\mathbb{Z}}
\def\QQ{\mathbb{Q}}
\def\II{\mathbb{I}}
\definecolor{bluu}{RGB}{7,29,66}
\newcommand{\framedhref}[2]{\href{#1}{\fcolorbox{bluu}{bluu}{\textcolor{white}{#2}}}}
\newtheorem{teorema}{Teorema}[section]
\newtheorem{lema}[teorema]{Lema}
\newtheorem{proposicion}[teorema]{Proposición}
\newtheorem{corolario}[teorema]{Corolario}
\newtheorem{definicion}[teorema]{Definición}
\newtheorem{ejemplo}[teorema]{Ejemplo}
\newtheorem{nota}[teorema]{Nota}
%-------------------------------------------------------------------------------------------------------
\hypersetup{pdfpagemode=FullScreen,colorlinks,linkcolor=,citecolor=black,urlcolor=links,pdftitle={pdftitle},pdfauthor={\talkauthor},pdfsubject={\talksubject},pdfkeywords={\talkkeywords}}
%-------------------------------------------------------------------------------------------------------
\setmainfont[Mapping={tex-text},Numbers={OldStyle},Ligatures=TeX]{Linux Biolinum
    O}
\setsansfont[Mapping={tex-text},Numbers={OldStyle},Ligatures=TeX]{Linux Libertine
    O}
\setmonofont[Mapping={tex-text},Numbers={OldStyle},Scale=0.8]{Pragmata
    Pro Mono}
%-------------------------------------------------------------------------------------------------------
\graphicspath{{/home/hsigrist/Dropbox/images/}}
\everymath{\displaystyle}
\AtBeginSection[]{\begin{frame}<beamer>\frametitle{Agenda}\tableofcontents[sectionstyle=show/hide,subsectionstyle=hide/show/hide,currentsection]\end{frame}\addtocounter{framenumber}{-1}}
%-------------------------------------------------------------------------------------------------------
\title{\talktitle}
\subtitle{\talksubtitle}
\author{\talkauthor}
\institute{\talkaffiliation}
\date{\footnotesize{\emph{\href{\talkblog}{\talkemail}}}}
%-------------------------------------------------------------------------------------------------------
\begin{document}
\begin{frame}
\titlepage
\end{frame}
%-------------------------------------------------------------------------------------------------------
\section{Sucesiones y series}
\begin{frame}
  \frametitle{Progresión aritmética}
  \begin{block}{Sumatoria Sigma}
    \begin{eqnarray}
      1+2+3+4+\cdots+100&=&\sum_{i=1}^{100}i\\
      2+4+6+8+\cdots+100&=&\sum_{i=1}^{50}2i\\
      3+6+9+12+\cdots+300&=&\sum_{i=1}^{100}3i\\
      50+51+52+\cdots+400&=&\sum_{i=50}^{400}i
    \end{eqnarray}
  \end{block}
\end{frame}
%-------------------------------------------------------------------------------------------------------
\begin{frame}
  \setcounter{equation}{0}
  \frametitle{Progresión aritmética}
  \begin{block}{Suma parcial}
    \begin{eqnarray}
      \sum_{i=1}^{100}i&=&\frac{100\cdot 101}{2}\\
      \sum_{i=1}^{50}2i&=&\\
      \sum_{i=1}^{100}3i&=&\\
      \sum_{i=50}^{40}i&=&
    \end{eqnarray}
  \end{block}
\end{frame}
%-------------------------------------------------------------------------------------------------------
\begin{frame}
  \frametitle{Progresión geométrica}
  \begin{block}{Sumatoria Sigma}
    \begin{eqnarray}
      2+4+8+16+\cdots+1024&=&\sum_{i=1}^{10}2^i\\
      3+9+27+81+\cdots+59049&=&\sum_{i=1}^{10}3^{i}\\
      3+6+9+12+\cdots+300&=&\sum_{i=1}^{100}3i\\
      50+51+52+\cdots+400&=&\sum_{i=50}^{400}i
    \end{eqnarray}
  \end{block}
\end{frame}
%-------------------------------------------------------------------------------------------------------
\begin{frame}
  \frametitle{Sumatorias combinadas}
  \begin{block}{Sumatoria Sigma}
    \begin{eqnarray}
      \sum_{i=1}^{4}\frac{2i}{2i-1}&=&\\
      \frac{\sum_{i=1}^{4}2i}{\sum_{i=1}^{4}2i-1}&=&
    \end{eqnarray}
  \end{block}
\end{frame}
%-------------------------------------------------------------------------------------------------------



\end{document}
%-------------------------------------------------------------------------------------------------------
% !TEX program = xelatex
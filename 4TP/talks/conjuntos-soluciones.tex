\documentclass[12pt,spanish,x11names]{beamer}
% -------------------------------------------------------------------------------------------------------
\usepackage{pgfpages}
% \setbeameroption{hide notes}
% \setbeameroption{show notes}
% \setbeameroption{show notes on second screen=right}
\usetheme{Hytex}
\setbeamertemplate{navigation symbols}{}
\usecolortheme[RGB={7,29,66}]{structure}
\usepackage{tcolorbox}
\usepackage{fourier}
\usepackage{float}
\usepackage{fontspec}
\usepackage{graphicx}
\usepackage{amssymb,amsmath}
\usepackage{polyglossia}
\setdefaultlanguage{spanish}
\usepackage[style=spanish]{csquotes}
\usepackage{pstricks-add}
\usepackage{tkz-euclide}
\usetkzobj{all}
\usepackage{pgf,tikz}
\usetikzlibrary{mindmap,trees,arrows}
\usepackage{siunitx}
\usepackage{xcolor}
\usepackage{booktabs}
\usepackage{marvosym}
\setbeamertemplate{caption}[numbered]
\usepackage{hyperref}
%-------------------------------------------------------------------------------------------------------
\def\talkclass{Presentación}
\def\talkcar{4TP}
\def\talkdate{\today}
\def\talkversion{}
\def\talktitle{Operatoria con conjuntos soluciones}
\def\talksubtitle{Álgebra de conjuntos en $\RR$}
\def\talkkeywords{uniones, intersecciones, representación gráfica, desigualdades}
\def\talksubject{geometría analítica}
\def\talkblog{https://hsigrist.github.io}
\def\talkpubpdf{}
\def\talkcopyright{\myauthor}
\def\talkaffiliation{Liceo Mixto Los Andes}
\def\talkauthor{Hans Sigrist}
\def\talkgrade{Lic. \& Mag. Matemática}
\def\talkemail{hsigrist@liceomixto.cl}
\definecolor{links}{HTML}{000000}
\def\NN{\mathbb{N}}
\def\RR{\mathbb{R}}
\def\ZZ{\mathbb{Z}}
\def\QQ{\mathbb{Q}}
\def\II{\mathbb{I}}
\definecolor{bluu}{RGB}{7,29,66}
\newcommand{\framedhref}[2]{\href{#1}{\fcolorbox{bluu}{bluu}{\textcolor{white}{#2}}}}
\newtheorem{teorema}{Teorema}[section]
\newtheorem{lema}[teorema]{Lema}
\newtheorem{proposicion}[teorema]{Proposición}
\newtheorem{corolario}[teorema]{Corolario}
\newtheorem{definicion}[teorema]{Definición}
\newtheorem{ejemplo}[teorema]{Ejemplo}
\newtheorem{nota}[teorema]{Nota}
%-------------------------------------------------------------------------------------------------------
\hypersetup{pdfpagemode=FullScreen,colorlinks,linkcolor=,citecolor=black,urlcolor=links,pdftitle={pdftitle},pdfauthor={\talkauthor},pdfsubject={\talksubject},pdfkeywords={\talkkeywords}}
%-------------------------------------------------------------------------------------------------------
\setmainfont[Mapping={tex-text},Numbers={OldStyle},Ligatures=TeX]{Linux Biolinum
    O}
\setsansfont[Mapping={tex-text},Numbers={OldStyle},Ligatures=TeX]{Linux Libertine
    O}
\setmonofont[Mapping={tex-text},Numbers={OldStyle},Ligatures=Rare,Scale=0.8]{Pragmata
    Pro Mono}
%-------------------------------------------------------------------------------------------------------
\graphicspath{{/home/hsigrist/Dropbox/images/}}
\everymath{\displaystyle}
\AtBeginSection[]{\begin{frame}<beamer>\frametitle{Agenda}\tableofcontents[sectionstyle=show/hide,subsectionstyle=hide/show/hide,currentsection]\end{frame}\addtocounter{framenumber}{-1}}
%-------------------------------------------------------------------------------------------------------
\title{\talktitle}
\subtitle{\talksubtitle}
\author{\talkauthor}
\institute{\talkaffiliation}
\date{\footnotesize{\emph{\href{\talkblog}{\talkemail}}}}
%-------------------------------------------------------------------------------------------------------
\begin{document}
\begin{frame}
\titlepage
\end{frame}
%-------------------------------------------------------------------------------------------------------
\section{Operatoria en conjuntos}
\begin{frame}
  \frametitle{Nuevos intervalos en $\RR$}
  \begin{exampleblock}{Problema}
  Considere los intervalos $A=\mathclose[ 1,5 \mathclose]$, $B=\mathopen]
  3,+\infty \mathopen[$. Determine $A\cup B$ y $A\cap B$.
     \begin{figure}[H]
      \centering
      \psset{xunit=.5cm, yunit=.5cm, yAxis=false}
      \begin{pspicture}(-1,0)(11,0)
        \psaxes[Dx=1, subticks=1]{<->}(0,0)(-1,0)(12,0)
        \psline[linewidth=2pt, linecolor=cyan]{*-*}(1,0)(5,0)
        \psline[linewidth=2pt, linecolor=red]{o->}(3,0)(13,0)
        \uput[-90](3,2){$A$}
        \uput[-90](10,2){$B$}
      \end{pspicture}	
    \end{figure}
     \pause
    \vspace{.4cm}
   \begin{figure}[H]
      \centering
      \psset{xunit=.5cm, yunit=.5cm, yAxis=false}
      \begin{pspicture}(-1,0)(11,0)
        \psaxes[Dx=1, subticks=1]{<->}(0,0)(-1,0)(12,0)
        \psline[linewidth=2pt, linecolor=green]{*->}(1,0)(13,0)
        \uput[-90](7,2){$A\cup B$}
      \end{pspicture}	
    \end{figure}
   \pause
    \vspace{.4cm}
   \begin{figure}[H]
      \centering
      \psset{xunit=1cm, yunit=1cm, yAxis=false}
      \begin{pspicture}(0,0)(4,0)
        \psaxes[Dx=1,subticks=1]{<->}(0,0)(-1,0)(6,0)
        \psline[linewidth=2pt,linecolor=blue]{o-*}(3,0)(5,0)
        \uput[-90](4,.8){$A\cap B$}
      \end{pspicture}	
    \end{figure}
    \vspace{1cm}
  \end{exampleblock}
\end{frame}
%-------------------------------------------------------------------------------------------------------
\section{Actividades}
\begin{frame}
  \frametitle{Actividades}
  \begin{exampleblock}{Determina las siguientes uniones e intersecciones de intervalos. Expresa tu
resultado como intervalo y represéntalo gráficamente en la recta real.}
\begin{enumerate}
\item $\mathclose[ 2,5 \mathclose[\cup \mathopen]3,18\mathclose[$ 
\item $\mathopen] -5,1 \mathopen]\cap \mathopen]1,7\mathclose[$ 
\item $\mathclose[ -\frac{7}{4},\frac{5}{3} \mathclose[\cup \mathopen] 0,+\infty\mathclose[$ 
\item $\mathclose[ -\frac{7}{4},\frac{5}{3} \mathclose[\cap \mathopen]
  0,+\infty\mathclose[$
  \item $\mathclose[ 0,1 \mathclose[\cap \left(\mathopen] -3,1 \mathclose[ \cap
      \mathclose[0,5 \mathopen] \right)$
\end{enumerate}
  \end{exampleblock}
\end{frame}
%-------------------------------------------------------------------------------------------------------
\begin{frame}
  \frametitle{Actividades}
  \begin{exampleblock}{Escribe una unión o intersección de intervalos cuyo conjunto solución esté representado en las
      siguientes figuras.}
    \vspace{1cm} 
       \begin{figure}[H]
      \centering
      \psset{xunit=1cm, yunit=1cm, yAxis=false}
      \begin{pspicture}(0,0)(4,0)
        \psaxes[Dx=1,subticks=1]{<->}(0,0)(-1,0)(5,0)
        \psline[linewidth=2pt,linecolor=blue]{o->}(3,0)(5,0)
      \end{pspicture}	
    \end{figure}

        \begin{figure}[H]
      \centering
      \psset{xunit=1.5cm, yunit=1.5cm, yAxis=false}
      \begin{pspicture}(-2,0)(2,0)
        \psaxes[Dx=1,subticks=1]{<->}(0,0)(-2,0)(2,0)
        \psline[linewidth=2pt,linecolor=green]{o-*}(-0.4,0)(0,0)
        \uput[-90](-0.4,.6){$-2/5$}
      \end{pspicture}	
    \end{figure}

      \begin{figure}[H]
      \centering
      \psset{xunit=.7cm, yunit=.7cm, yAxis=false}
      \begin{pspicture}(-1,0)(8,0)
        \psaxes[Dx=1,subticks=1]{<->}(0,0)(-1,0)(8,0)
        \psline[linewidth=2pt,linecolor=blue]{*-*}(1.5,0)(6.33,0)
        \uput[-90](1.5,1.3){$3/2$}
        \uput[-90](6.3,1.3){$19/3$}
      \end{pspicture}	
    \end{figure}

    \vspace{1cm} 
  \end{exampleblock}
\end{frame}



%-------------------------------------------------------------------------------------------------------
\begin{frame}[c]\frametitle{Apéndice}
\centering\decofourleft\quad\decofourright

\textbf{\emph {¡Carpe diem!}}

Una copia del presente trabajo, se encuentra en el enlace \framedhref{\talkpubpdf}{\talktitle}.
\end{frame}
\end{document}
%-------------------------------------------------------------------------------------------------------
%!TEX program = xelatex
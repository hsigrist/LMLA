\documentclass[12pt,spanish,x11names]{beamer}
% -------------------------------------------------------------------------------------------------------
\usepackage{pgfpages}
% \setbeameroption{hide notes}
% \setbeameroption{show notes}
% \setbeameroption{show notes on second screen=right}
\usetheme{Hytex}
\setbeamertemplate{navigation symbols}{}
\usecolortheme[RGB={7,29,66}]{structure}
\usepackage{tcolorbox}
\usepackage{fourier}
\usepackage{float}
\usepackage{fontspec}
\usepackage{graphicx}
\usepackage{amssymb,amsmath}
\usepackage{polyglossia}
\setdefaultlanguage{spanish}
\usepackage[style=spanish]{csquotes}
\usepackage{pstricks-add}
\usepackage{tkz-euclide}
\usetkzobj{all}
\usepackage{pgf,tikz}
\usetikzlibrary{mindmap,trees,arrows}
\usepackage{siunitx}
\usepackage{xcolor}
\usepackage{booktabs}
\usepackage{marvosym}
\setbeamertemplate{caption}[numbered]
\usepackage{hyperref}
%-------------------------------------------------------------------------------------------------------
\def\talkclass{Presentación}
\def\talkcar{4TP}
\def\talkdate{\today}
\def\talkversion{}
\def\talktitle{Inecuaciones lineales}
\def\talksubtitle{Inecuaciones con una incógnita}
\def\talkkeywords{inecuaciones}
\def\talksubject{geometría analítica}
\def\talkblog{https://hsigrist.github.io}
\def\talkpubpdf{https://github.com/hsigrist/LMLA/blob/master/4TP/talks/inecuaciones-lineales.pdf}
\def\talkcopyright{\myauthor}
\def\talkaffiliation{Liceo Mixto Los Andes}
\def\talkauthor{Hans Sigrist}
\def\talkgrade{Lic. \& Mag. Matemática}
\def\talkemail{hsigrist@liceomixto.cl}
\definecolor{links}{HTML}{000000}
\def\NN{\mathbb{N}}
\def\RR{\mathbb{R}}
\def\ZZ{\mathbb{Z}}
\def\QQ{\mathbb{Q}}
\def\II{\mathbb{I}}
\definecolor{bluu}{RGB}{7,29,66}
\newcommand{\framedhref}[2]{\href{#1}{\fcolorbox{bluu}{bluu}{\textcolor{white}{#2}}}}
\newtheorem{teorema}{Teorema}[section]
\newtheorem{lema}[teorema]{Lema}
\newtheorem{proposicion}[teorema]{Proposición}
\newtheorem{corolario}[teorema]{Corolario}
\newtheorem{definicion}[teorema]{Definición}
\newtheorem{ejemplo}[teorema]{Ejemplo}
\newtheorem{nota}[teorema]{Nota}
%-------------------------------------------------------------------------------------------------------
\hypersetup{pdfpagemode=FullScreen,colorlinks,linkcolor=,citecolor=black,urlcolor=links,pdftitle={pdftitle},pdfauthor={\talkauthor},pdfsubject={\talksubject},pdfkeywords={\talkkeywords}}
%-------------------------------------------------------------------------------------------------------
\setmainfont[Mapping={tex-text},Ligatures=TeX]{Linux Biolinum
    O}
\setsansfont[Mapping={tex-text},Ligatures=TeX]{Linux Libertine
    O}
\setmonofont[Mapping={tex-text},Numbers={OldStyle},Ligatures=Rare,Scale=0.8]{Pragmata
    Pro Mono}
%-------------------------------------------------------------------------------------------------------
\graphicspath{{/home/hsigrist/Dropbox/images/}}
\everymath{\displaystyle}
\AtBeginSection[]{\begin{frame}<beamer>\frametitle{Agenda}\tableofcontents[sectionstyle=show/hide,subsectionstyle=hide/show/hide,currentsection]\end{frame}\addtocounter{framenumber}{-1}}
%-------------------------------------------------------------------------------------------------------
\title{\talktitle}
\subtitle{\talksubtitle}
\author{\talkauthor}
\institute{\talkaffiliation}
\date{\footnotesize{\emph{\href{\talkblog}{\talkemail}}}}
%-------------------------------------------------------------------------------------------------------
\begin{document}
\begin{frame}
\titlepage
\end{frame}
%-------------------------------------------------------------------------------------------------------
\section{Inecuaciones}
\begin{frame}
  \frametitle{Problematización}
  \begin{exampleblock}{Si un joven es $22$ años menor que su padre y $48$ años menor que su abuelo,
¿a partir de qué edad la suma de los años que tienen él y su padre será mayor
que la edad de su abuelo?}
\pause
    Si definimos como $x$ la edad del joven, entonces la edad de su padre y su abuelo
    serán $x+22$ y $x+48$, respectivamente.
    \begin{align*}
      x+x+22&>x+48\\
      2x+22&>x+48\\
      x+22&>48\\
      x&>26
    \end{align*}
  \end{exampleblock}
\end{frame}
%-------------------------------------------------------------------------------------------------------
\begin{frame}
  \frametitle{Definición}
  \begin{block}{Inecuación}
    \begin{itemize}
    \item Una \textbf{inecuación} es una \alert{desigualdad} que tiene una o más incógnitas. Para
    resolverla, debemos encontrar todos los valores de las incógnitas que hacen
    verdadera la desigualdad.
    \item El conjunto solución de una inecuación con una incógnita se puede
    representar mediante un \textbf{intervalo}, o bien, \textbf{gráficamente} en la recta numérica.
    \end{itemize}
  \end{block}
\end{frame}
%-------------------------------------------------------------------------------------------------------
\begin{frame}
  \frametitle{Ejemplo}
  \begin{exampleblock}{Determina el conjunto solución de $x-2(x-3)>0$ y
      represéntalo gráficamente en la recta real.}
\pause
\begin{align*}
  x-2x+6&>0\\
  -x+6&>0\\
  6&>x\\
  \Leftrightarrow x&<6
\end{align*}
\begin{figure}[H]
  \centering
  \psset{xunit=1cm, yunit=1cm, yAxis=false}
  \begin{pspicture}(-1,0)(8,0)
    \psaxes[Dx=1,subticks=1]{<->}(0,0)(-1,0)(8,0)
    \psline[linewidth=2pt,linecolor=blue]{<-o}(-1,0)(6,0)
  \end{pspicture}	
\end{figure}
\vspace{1cm}
  \end{exampleblock}
\end{frame}
%-------------------------------------------------------------------------------------------------------
\section{Actividades}
\begin{frame}
  \frametitle{Actividades}
  \begin{exampleblock}{Resuelve las siguientes inecuaciones lineales con una incógnita. Represente las soluciones como intervalo y gráficamente.}
\begin{enumerate}
\item $3+4x<51$
\item $-3x>18$
\item $(x+2)(x+1)\geq(x+3)²$
\item $x+3(x-5)<6-4(2-3x)$
\item $3\leq\frac{5x-1}{4}$
\item $\frac{x}{4}+\frac{x}{2}>\frac{x}{3}+4$
\end{enumerate}
\end{exampleblock}
\end{frame}
%-------------------------------------------------------------------------------------------------------
\begin{frame}
  \frametitle{Actividades}
  \begin{exampleblock}{Resolución de problemas. Plantee la inecuación en cada
      caso y represente su solución como conjunto, intervalo y gráficamente.}
    \begin{enumerate}
    \item La suma entre un número natural y su sucesor es inferior a $12$. ¿Qué valores
      puede adoptar tal número?
    \item Una fábrica paga a sus vendedores $\$880$ por artículo vendido, más
      una cantidad fija de $\$286100$. Si un vendedor quiere que su sueldo sea
      superior a $\$340000$, ¿Cuántos artículos debe vender como mínimo?
    \item La suma de tres números consecutivos es mayor que $60$. ¿Cuál es el
      menor valor que podría adoptar el número mayor?
    \end{enumerate}
  \end{exampleblock}
\end{frame}
%-------------------------------------------------------------------------------------------------------
\begin{frame}[c]\frametitle{Apéndice}
\centering\decofourleft\quad\decofourright

\textbf{\emph {¡Carpe diem!}}

Una copia del presente trabajo, se encuentra en el enlace \framedhref{\talkpubpdf}{\talktitle}.
\end{frame}
\end{document}
%-------------------------------------------------------------------------------------------------------
%!TEX program = xelatex
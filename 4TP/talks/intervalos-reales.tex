\documentclass[12pt,handout,spanish,x11names]{beamer}
%-------------------------------------------------------------------------------------------------------
\usepackage{pgfpages}
% \setbeameroption{hide notes}
% \setbeameroption{show notes}
% \setbeameroption{show notes on second screen=right}
\usetheme{Hytex}
\setbeamertemplate{navigation symbols}{}
\usecolortheme[RGB={7,29,66}]{structure}
\usepackage{tcolorbox}
\usepackage{fourier}
\usepackage{float}
\usepackage{fontspec}
\usepackage{graphicx}
\usepackage{amssymb,amsmath}
\usepackage{polyglossia}
\setdefaultlanguage{spanish}
\usepackage[style=spanish]{csquotes}
\usepackage{pstricks-add}
\usepackage{tkz-euclide}
\usetkzobj{all}
\usepackage{pgf,tikz}
\usetikzlibrary{mindmap,trees,arrows}
\usepackage{siunitx}
\usepackage{xcolor}
\usepackage{booktabs}
\usepackage{marvosym}
\setbeamertemplate{caption}[numbered]
\usepackage{hyperref}
%-------------------------------------------------------------------------------------------------------
\def\talkclass{Presentación}
\def\talkcar{4TP}
\def\talkdate{\today}
\def\talkversion{}
\def\talktitle{Intervalos Reales}
\def\talksubtitle{Representación gráfica de $\RR$. Operatoria con intervalos.}
\def\talkkeywords{orden en $\RR$, desigualdades, inecuaciones}
\def\talksubject{geometría analítica}
\def\talkblog{https://hsigrist.github.io}
\def\talkpubpdf{https://github.com/hsigrist/LMLA/blob/master/4TP/talks/intervalos-reales.pdf}
\def\talkcopyright{\myauthor}
\def\talkaffiliation{Liceo Mixto Los Andes}
\def\talkauthor{Hans Sigrist}
\def\talkgrade{Lic. \& Mag. Matemática}
\def\talkemail{hsigrist@liceomixto.cl}
\definecolor{links}{HTML}{000000}
\def\NN{\mathbb{N}}
\def\RR{\mathbb{R}}
\def\ZZ{\mathbb{Z}}
\def\QQ{\mathbb{Q}}
\def\II{\mathbb{I}}
\definecolor{bluu}{RGB}{7,29,66}
\newcommand{\framedhref}[2]{\href{#1}{\fcolorbox{bluu}{bluu}{\textcolor{white}{#2}}}}
\newtheorem{teorema}{Teorema}[section]
\newtheorem{lema}[teorema]{Lema}
\newtheorem{proposicion}[teorema]{Proposición}
\newtheorem{corolario}[teorema]{Corolario}
\newtheorem{definicion}[teorema]{Definición}
\newtheorem{ejemplo}[teorema]{Ejemplo}
\newtheorem{nota}[teorema]{Nota}
%-------------------------------------------------------------------------------------------------------
\hypersetup{pdfpagemode=FullScreen,colorlinks,linkcolor=,citecolor=black,urlcolor=links,pdftitle={pdftitle},pdfauthor={\talkauthor},pdfsubject={\talksubject},pdfkeywords={\talkkeywords}}
%-------------------------------------------------------------------------------------------------------
\setmainfont[Mapping={tex-text},Numbers={OldStyle},Ligatures=TeX]{Linux Biolinum
    O}
\setsansfont[Mapping={tex-text},Numbers={OldStyle},Ligatures=TeX]{Linux Libertine
    O}
\setmonofont[Mapping={tex-text},Numbers={OldStyle},Ligatures=Rare,Scale=0.8]{Pragmata
    Pro Mono}
%-------------------------------------------------------------------------------------------------------
\graphicspath{{/home/hsigrist/Dropbox/images/}}
\everymath{\displaystyle}
\AtBeginSection[]{\begin{frame}<beamer>\frametitle{Agenda}\tableofcontents[sectionstyle=show/hide,subsectionstyle=hide/show/hide,currentsection]\end{frame}\addtocounter{framenumber}{-1}}
%-------------------------------------------------------------------------------------------------------
\title{\talktitle}
\subtitle{\talksubtitle}
\author{\talkauthor}
\institute{\talkaffiliation}
\date{\footnotesize{\emph{\href{\talkblog}{\talkemail}}}}
%-------------------------------------------------------------------------------------------------------
\begin{document}
\begin{frame}
\titlepage
\end{frame}
%-------------------------------------------------------------------------------------------------------
\section{Expresar información por medio de desigualdades}
\begin{frame}
  \frametitle{Conexiones con la industria}
  \begin{exampleblock}{Expresar por medio de desigualdades}
    \begin{description}
    \item[Medio Ambiente] Se considera que la calidad del aire es ``regular'' si el índice de
      calidad del aire por material particulado (ICAP) es superior a $100$ y
      menor o igual a $200$.
    \item[Medicina] En un examen que mide la cantidad de glucosa en la sangre de
      una persona adulta, se consideran normales los valores que van de $64$ a
      $110 mg/dL$ (miligramos por decilitro).
    \item[Física] La longitud de onda
      de la luz visible es superior a $380 nm$ y menor o igual a $780 nm$.
    \end{description}
  \end{exampleblock}
\end{frame}
%-------------------------------------------------------------------------------------------------------
\begin{frame}
  \frametitle{Modelos suyos}
  \begin{exampleblock}{}
    \begin{enumerate}
    \item $r<6$
    \item $P\geq 4.95$
    \item $R<4.45$
    \item $m<n-15$
    \item $a+b<132$
    \end{enumerate}
  \end{exampleblock}
\end{frame}
%-------------------------------------------------------------------------------------------------------
\begin{frame}
  \frametitle{Definición}
  \begin{block}{Desigualdad}
    Se denomina \textbf{desigualdad} a toda relación de orden que se establece entre números reales u otras
    expresiones matemáticas, mediante la comparación:
    \begin{itemize}
    \item ``menor que'' ($<$),
    \item ``menor o igual que'' ($\leq$),
    \item ``mayor que'' ($>$) o
    \item ``mayor o igual que'' ($\geq$).
    \end{itemize}
  \end{block}
\end{frame}
%-------------------------------------------------------------------------------------------------------
\section{Representación gráfica de $\RR$}
\begin{frame}
  \frametitle{Intervalos en $\RR$}
  \begin{exampleblock}{Intervalo cerrado}
    \vspace{1cm}
  \begin{figure}[H]
    \centering
    Representación gráfica: 
    \psset{xunit=1cm, yunit=1cm, yAxis=false}
    \begin{pspicture}(-1,0)(7,0)
      \psaxes[Dx=1, subticks=1]{<->}(0,0)(-1,0)(6,0)
      \psline[linewidth=2pt, linecolor=cyan]{*-*}(2,0)(4,0)
    \end{pspicture}	
  \end{figure}
  \vspace{1cm}
  \begin{equation*}
   \text{Notación conjunto: } \left\{x\in\RR:2\leq x\leq 4\right\} 
 \end{equation*}
 \vspace{.5cm}
  \begin{equation*}
   \text{Notación intervalo: } \left[ 2,4 \right] 
  \end{equation*}
  \end{exampleblock}
\end{frame}
%-------------------------------------------------------------------------------------------------------
\begin{frame}
  \frametitle{Intervalos en $\RR$}
  \begin{exampleblock}{Intervalo abierto}
    \vspace{1cm}
  \begin{figure}[H]
    \centering
    Representación gráfica: 
    \psset{xunit=1cm, yunit=1cm, yAxis=false}
    \begin{pspicture}(-1,0)(7,0)
      \psaxes[Dx=1, subticks=1]{<->}(0,0)(-1,0)(6,0)
      \psline[linewidth=2pt, linecolor=cyan]{o-o}(2,0)(4,0)
    \end{pspicture}	
  \end{figure}
  \vspace{1cm}
  \begin{equation*}
   \text{Notación conjunto: } \left\{x\in\RR:2< x< 4\right\} 
 \end{equation*}
 \vspace{.5cm}
  \begin{equation*}
    \text{Notación intervalo: } \mathopen] 2,4 \mathclose[
      \end{equation*}
  \end{exampleblock}
\end{frame}
%-------------------------------------------------------------------------------------------------------
\begin{frame}
  \frametitle{Intervalos en $\RR$}
  \begin{exampleblock}{Intervalo semi-abierto}
    \vspace{1cm}
  \begin{figure}[H]
    \centering
    Representación gráfica: 
    \psset{xunit=1cm, yunit=1cm, yAxis=false}
    \begin{pspicture}(-1,0)(7,0)
      \psaxes[Dx=1, subticks=1]{<->}(0,0)(-1,0)(6,0)
      \psline[linewidth=2pt, linecolor=cyan]{*-o}(2,0)(4,0)
    \end{pspicture}	
  \end{figure}
  \vspace{1cm}
  \begin{equation*}
   \text{Notación conjunto: } \left\{x\in\RR:2\leq x< 4\right\} 
 \end{equation*}
 \vspace{.5cm}
  \begin{equation*}
    \text{Notación intervalo: } \mathclose[ 2,4 \mathclose[
      \end{equation*}
  \end{exampleblock}
\end{frame}
%-------------------------------------------------------------------------------------------------------
\begin{frame}
  \frametitle{Intervalos en $\RR$}
  \begin{exampleblock}{Intervalo semi-abierto}
    \vspace{1cm}
  \begin{figure}[H]
    \centering
    Representación gráfica: 
    \psset{xunit=1cm, yunit=1cm, yAxis=false}
    \begin{pspicture}(-1,0)(7,0)
      \psaxes[Dx=1, subticks=1]{<->}(0,0)(-1,0)(6,0)
      \psline[linewidth=2pt, linecolor=cyan]{o-*}(2,0)(4,0)
    \end{pspicture}	
  \end{figure}
  \vspace{1cm}
  \begin{equation*}
   \text{Notación conjunto: } \left\{x\in\RR:2< x\leq 4\right\} 
 \end{equation*}
 \vspace{.5cm}
  \begin{equation*}
    \text{Notación intervalo: } \mathopen] 2,4 \mathclose]
      \end{equation*}
  \end{exampleblock}
\end{frame}
%-------------------------------------------------------------------------------------------------------
\begin{frame}
  \frametitle{Intervalos en $\RR$}
  \begin{exampleblock}{Intervalo no acotado o infinito}
    \vspace{1cm}
  \begin{figure}[H]
    \centering
    Representación gráfica: 
    \psset{xunit=1cm, yunit=1cm, yAxis=false}
    \begin{pspicture}(-1,0)(7,0)
      \psaxes[Dx=1, subticks=1]{<->}(0,0)(-1,0)(6,0)
      \psline[linewidth=2pt, linecolor=cyan]{*->}(2,0)(7,0)
    \end{pspicture}	
  \end{figure}
  \vspace{1cm}
  \begin{equation*}
   \text{Notación conjunto: } \left\{x\in\RR:x\geq 2 \right\} 
 \end{equation*}
 \vspace{.5cm}
  \begin{equation*}
    \text{Notación intervalo: } \mathclose[ 2,+\infty \mathopen[
      \end{equation*}
  \end{exampleblock}
\end{frame}
%-------------------------------------------------------------------------------------------------------
\begin{frame}
  \frametitle{Intervalos en $\RR$}
  \begin{exampleblock}{Intervalo no acotado o infinito}
    \vspace{1cm}
  \begin{figure}[H]
    \centering
    Representación gráfica: 
    \psset{xunit=1cm, yunit=1cm, yAxis=false}
    \begin{pspicture}(-1,0)(7,0)
      \psaxes[Dx=1, subticks=1]{<->}(0,0)(-1,0)(6,0)
      \psline[linewidth=2pt, linecolor=cyan]{o->}(2,0)(7,0)
    \end{pspicture}	
  \end{figure}
  \vspace{1cm}
  \begin{equation*}
   \text{Notación conjunto: } \left\{x\in\RR:x> 2\right\} 
 \end{equation*}
 \vspace{.5cm}
  \begin{equation*}
    \text{Notación intervalo: } \mathopen] 2,+\infty \mathopen[
      \end{equation*}
  \end{exampleblock}
\end{frame}
%-------------------------------------------------------------------------------------------------------
\begin{frame}
  \frametitle{Intervalos en $\RR$}
  \begin{exampleblock}{Intervalo no acotado o infinito}
    \vspace{1cm}
  \begin{figure}[H]
    \centering
    Representación gráfica: 
    \psset{xunit=1cm, yunit=1cm, yAxis=false}
    \begin{pspicture}(-1,0)(7,0)
      \psaxes[Dx=1, subticks=1]{<->}(0,0)(-1,0)(6,0)
      \psline[linewidth=2pt, linecolor=cyan]{<-*}(-1,0)(4,0)
    \end{pspicture}	
  \end{figure}
  \vspace{1cm}
  \begin{equation*}
   \text{Notación conjunto: } \left\{x\in\RR:x\leq 4\right\} 
 \end{equation*}
 \vspace{.5cm}
  \begin{equation*}
    \text{Notación intervalo: } \mathopen] -\infty,4 \mathclose]
      \end{equation*}
  \end{exampleblock}
\end{frame}
%-------------------------------------------------------------------------------------------------------
\begin{frame}
  \frametitle{Intervalos en $\RR$}
  \begin{exampleblock}{Intervalo no acotado o infinito}
    \vspace{1cm}
  \begin{figure}[H]
    \centering
    Representación gráfica: 
    \psset{xunit=1cm, yunit=1cm, yAxis=false}
    \begin{pspicture}(-1,0)(7,0)
      \psaxes[Dx=1, subticks=1]{<->}(0,0)(-1,0)(6,0)
      \psline[linewidth=2pt, linecolor=cyan]{<-o}(-1,0)(4,0)
    \end{pspicture}	
  \end{figure}
  \vspace{1cm}
  \begin{equation*}
   \text{Notación conjunto: } \left\{x\in\RR:x<4\right\} 
 \end{equation*}
 \vspace{.5cm}
  \begin{equation*}
    \text{Notación intervalo: } \mathopen] -\infty,4 \mathclose[
      \end{equation*}
  \end{exampleblock}
\end{frame}
%-------------------------------------------------------------------------------------------------------
\section{Actividades}
\begin{frame}
  \frametitle{Actividades}
  \begin{exampleblock}{Encuentra tres números que pertenezcan a cada uno de los intervalos}
    \begin{enumerate}
    \item $\mathopen]0,1\mathopen[$
    \item $\mathopen]1.41,\sqrt{2}\mathopen[$
    \item $\mathopen]\pi,4\mathclose]$
    \item $\mathopen]-0.001,0\mathopen[$
    \item $\mathopen]\sqrt{2},\sqrt{3}\mathopen[$
    \end{enumerate}
  \end{exampleblock}
\end{frame}
%-------------------------------------------------------------------------------------------------------
\begin{frame}
  \frametitle{Actividades}
  \begin{exampleblock}{Exprese como intervalo y represente gráficamente}
    \begin{enumerate}
      \item $\left\{ x\in\RR/-\sqrt{3}<x \right\}$
      \item $\left\{ x\in\RR/\frac{1}{5}<x\leq1.33 \right\}$
      \item $\left\{ x\in\RR/0<x\leq0.5 \right\}$
      \item $\left\{ x\in\RR/x\leq-3 \right\}$
      \item $\left\{ x\in\RR/x>\frac{4}{5} \right\}$
    \end{enumerate}
  \end{exampleblock}
\end{frame}
%-------------------------------------------------------------------------------------------------------
\begin{frame}
  \frametitle{Desafío}
  \begin{alertblock}{Considere los números $0$, $\pi$, $\sqrt{2}$ y $\frac{3}{4}$}
    \begin{enumerate}
    \item Encuentra un intervalo que contenga todos estos números.
    \item Encuentra un intervalo que no contenga ninguno de ellos.
      \item Para cada número, encuentra un intervalo cerrado que lo contenga y cuyos extremos sean números
        enteros consecutivos.
    \end{enumerate}
  \end{alertblock}
\end{frame}
%-------------------------------------------------------------------------------------------------------
\begin{frame}[c]\frametitle{Apéndice}
\centering\decofourleft\quad\decofourright

\textbf{\emph {¡Carpe diem!}}

Una copia del presente trabajo, se encuentra en el enlace \framedhref{\talkpubpdf}{\talktitle}.
\end{frame}
\end{document}
%-------------------------------------------------------------------------------------------------------
%!TEX program = xelatex
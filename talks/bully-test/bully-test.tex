\documentclass[12pt,spanish,x11names,svgnames]{beamer}
%-------------------------------------------------------------------------------------------------------
\usetheme{Hytex}
\usepackage{pgfpages}
%\setbeameroption{show notes on second screen=right}
\setbeamertemplate{navigation symbols}{}
\usecolortheme[RGB={20,13,67}]{structure}
\usepackage{tcolorbox}
\usepackage{fourier}
\usepackage{float}
\usepackage{fontspec}
\usepackage{graphicx}
\usepackage{amssymb,amsmath}
\usepackage{polyglossia}
\setdefaultlanguage{spanish}
\usepackage[style=spanish]{csquotes}
\usepackage{pstricks-add}
\usepackage{tkz-euclide}
\usetkzobj{all}
\usepackage{pgf,tikz}
\usetikzlibrary{mindmap,trees,arrows}
\usepackage{siunitx}
\usepackage{xcolor}
\usepackage{booktabs}
\usepackage{marvosym}
\setbeamertemplate{caption}[numbered]
\usepackage{hyperref}
%-------------------------------------------------------------------------------------------------------
\def\talkclass{Presentación}
\def\talkcar{Comunidad escolar}
\def\talkdate{\today}
\def\talkversion{TLK-1002}
\def\talktitle{bully-test}
\def\talksubtitle{¿Qué tipo de testigo eres?}
\def\talkkeywords{bullying, bully-test, acoso escolar}
\def\talksubject{acoso escolar}
\def\talkblog{https://hsigrist.github.io}
\def\talkpubpdf{https://www.dropbox.com/s/627vvg7phl1nyoe/bully-test.pdf?dl=0}
\def\talkcopyright{\myauthor}
\def\talkaffiliation{Liceo Mixto Los Andes}
\def\talkauthor{Hans Sigrist}
\def\talkgrade{Lic. \& Mag. Matemática}
\def\talkemail{hsigrist@liceomixto.cl}
\definecolor{links}{HTML}{000000}
\def\NN{\mathbb{N}}
\def\RR{\mathbb{R}}
\def\ZZ{\mathbb{Z}}
\def\QQ{\mathbb{Q}}
\def\II{\mathbb{I}}
\definecolor{bluu}{RGB}{7,29,66}
\definecolor{scarlet}{RGB}{255,36,0}
\newcommand{\framedhref}[2]{\href{#1}{\fcolorbox{bluu}{bluu}{\textcolor{white}{#2}}}}
\newtheorem{teorema}{Teorema}[section]
\newtheorem{lema}[teorema]{Lema}
\newtheorem{proposicion}[teorema]{Proposición}
\newtheorem{corolario}[teorema]{Corolario}
\newtheorem{definicion}[teorema]{Definición}
\newtheorem{ejemplo}[teorema]{Ejemplo}
\newtheorem{nota}[teorema]{Nota}
%-------------------------------------------------------------------------------------------------------
\hypersetup{pdfpagemode=FullScreen,colorlinks,linkcolor=,citecolor=black,urlcolor=links,pdftitle={pdftitle},pdfauthor={\talkauthor},pdfsubject={\talksubject},pdfkeywords={\talkkeywords}}
%-------------------------------------------------------------------------------------------------------
\setromanfont[Mapping={tex-text},Numbers={OldStyle},Ligatures=TeX]{Linux
  Biolinum O}
\setsansfont[Mapping={tex-text},Numbers={OldStyle},Ligatures=TeX]{Linux Biolinum
  O}
\setmonofont[Mapping={tex-text},Numbers={OldStyle},Ligatures=Rare,Scale=0.8]{Pragmata
  Pro Mono}
%-------------------------------------------------------------------------------------------------------
\graphicspath{{/home/hsigrist/LMLA/2018/talks/test-bullying/}}
\everymath{\displaystyle}
\AtBeginSection[]{\begin{frame}<beamer>\frametitle{Agenda}\tableofcontents[sectionstyle=show/hide,subsectionstyle=hide/show/hide,currentsection]\end{frame}\addtocounter{framenumber}{-1}}
%-------------------------------------------------------------------------------------------------------
\title{\talktitle}
\subtitle{\talksubtitle}
\author{\talkauthor}
\institute{\talkaffiliation}
\date{\footnotesize{\emph{\href{\talkblog}{\talkemail}}}}
%-------------------------------------------------------------------------------------------------------
\begin{document}
\begin{frame}
\titlepage
\end{frame}
%-------------------------------------------------------------------------------------------------------
\section{Emoji}
\begin{frame}
  \frametitle{Emoji soy testigo}
  \begin{figure}[H]
    \centering
    \includegraphics[scale=.35]{emoji}
  \end{figure}
\end{frame}
%-------------------------------------------------------------------------------------------------------
\begin{frame}
  \frametitle{Emoji soy testigo}
  \begin{figure}[H]
    \centering
    \includegraphics[scale=.51]{eyes}
  \end{figure}
\end{frame}

\section{Bully-Test}
\begin{frame}[c]\frametitle{¿Qué tipo de testigo eres?}
  \begin{block}{Pregunta 1}
    Un compañero de clase publicó un video de él mismo cantando en línea y la
    gente está dejando comentarios negativos sobre él. ¿Haces algo al respecto?
  \end{block}
  \pause
  \begin{exampleblock}{Si}
    Hablar y hacer saber a los demás que no es bueno burlarse de las
    personas, puede hacer una gran diferencia. Intenta publicar Emoji️ testigo
    para demostrar que no te gustan los malos comentarios.
  \end{exampleblock}
  \pause
  \begin{alertblock}{No}
    Es difícil saber cómo responder al acoso cibernético. Defender a otro niño
    no siempre es fácil y requiere coraje. Una forma es publicar el Emoji
    testigo en un mensaje hiriente para mostrar tu repudio.
  \end{alertblock}
\end{frame}
%-------------------------------------------------------------------------------------------------------
%-------------------------------------------------------------------------------------------------------
%-------------------------------------------------------------------------------------------------------
%-------------------------------------------------------------------------------------------------------
%-------------------------------------------------------------------------------------------------------
%-------------------------------------------------------------------------------------------------------
%-------------------------------------------------------------------------------------------------------
%-------------------------------------------------------------------------------------------------------
%-------------------------------------------------------------------------------------------------------
%-------------------------------------------------------------------------------------------------------
%-------------------------------------------------------------------------------------------------------
%-------------------------------------------------------------------------------------------------------
%-------------------------------------------------------------------------------------------------------
%-------------------------------------------------------------------------------------------------------
%-------------------------------------------------------------------------------------------------------
%-------------------------------------------------------------------------------------------------------
%-------------------------------------------------------------------------------------------------------
%-------------------------------------------------------------------------------------------------------
%-------------------------------------------------------------------------------------------------------
%-------------------------------------------------------------------------------------------------------
%-------------------------------------------------------------------------------------------------------
%-------------------------------------------------------------------------------------------------------
\begin{frame}[c]\frametitle{Apéndice}
\centering\decofourleft\quad\decofourright

\textbf{\emph {Saludos a todas/os, ¡Muchas gracias!}}

Una copia del presente trabajo, se encuentra en el enlace \framedhref{\talkpubpdf}{\talktitle}.
\end{frame}
\end{document}
%-------------------------------------------------------------------------------------------------------
% !TEX program = xelatex
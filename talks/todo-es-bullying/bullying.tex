\documentclass[12pt,spanish,x11names,svgnames]{beamer}
%-------------------------------------------------------------------------------------------------------
\usetheme{Hytex}
\setbeamertemplate{navigation symbols}{}
\usecolortheme[RGB={7,29,66}]{structure}
\usepackage{tcolorbox}
\usepackage{fourier}
\usepackage{fontspec}
\usepackage{graphicx}
\usepackage{amssymb,amsmath}
\usepackage{polyglossia}
\setdefaultlanguage{spanish}
\usepackage[style=spanish]{csquotes}
\usepackage{pstricks-add}
\usepackage{tkz-euclide}
\usetkzobj{all}
\usepackage{pgf,tikz}
\usetikzlibrary{mindmap,trees,arrows}
\usepackage{siunitx}
\usepackage{xcolor}
\usepackage{booktabs}
\usepackage{marvosym}
\setbeamertemplate{caption}[numbered]
\usepackage{hyperref}
%-------------------------------------------------------------------------------------------------------
\def\talkclass{Presentación}
\def\talkcar{Comunidad Escolar}
\def\talkdate{\today}
\def\talkversion{TLK-1001}
\def\talktitle{¿Todo es bullying?}
\def\talksubtitle{Una visión crítica}
\def\talkkeywords{bullying, pensamiento crítico}
\def\talksubject{maltrato escolar}
\def\talkblog{https://hsigrist.github.io}
\def\talkpubpdf{copyurlfromrepository}
\def\talkcopyright{\myauthor}
\def\talkaffiliation{Liceo Mixto Los Andes}
\def\talkauthor{Hans Sigrist}
\def\talkgrade{Lic. \& Mag. Matemática}
\def\talkemail{hsigrist@liceomixto.cl}
\definecolor{links}{HTML}{000000}
\def\NN{\mathbb{N}}
\def\RR{\mathbb{R}}
\def\ZZ{\mathbb{Z}}
\def\QQ{\mathbb{Q}}
\def\II{\mathbb{I}}
\definecolor{bluu}{RGB}{7,29,66}
\newcommand{\framedhref}[2]{\href{#1}{\fcolorbox{bluu}{bluu}{\textcolor{white}{#2}}}}
\newtheorem{teorema}{Teorema}[section]
\newtheorem{lema}[teorema]{Lema}
\newtheorem{proposicion}[teorema]{Proposición}
\newtheorem{corolario}[teorema]{Corolario}
\newtheorem{definicion}[teorema]{Definición}
\newtheorem{ejemplo}[teorema]{Ejemplo}
\newtheorem{nota}[teorema]{Nota}
%-------------------------------------------------------------------------------------------------------
\hypersetup{colorlinks,linkcolor=,citecolor=black,urlcolor=links,pdftitle={pdftitle},pdfauthor={\talkauthor},pdfsubject={\talksubject},pdfkeywords={\talkkeywords}}
%-------------------------------------------------------------------------------------------------------
\setromanfont[Mapping={tex-text},Numbers={OldStyle},Ligatures=TeX]{Linux Biolinum
    O}
\setsansfont[Mapping={tex-text},Numbers={OldStyle},Ligatures=TeX]{Linux Biolinum
    O}
\setmonofont[Mapping={tex-text},Numbers={OldStyle},Ligatures=Rare,Scale=0.8]{Pragmata
    Pro Mono}
%-------------------------------------------------------------------------------------------------------
\graphicspath{{/home/hsigrist/Dropbox/images/}}
\everymath{\displaystyle}
\AtBeginSection[]{\begin{frame}<beamer>\frametitle{Agenda}\tableofcontents[sectionstyle=show/hide,subsectionstyle=hide/show/hide,currentsection]\end{frame}\addtocounter{framenumber}{-1}}
%-------------------------------------------------------------------------------------------------------
\title{\talktitle}
\subtitle{\talksubtitle}
\author{\talkauthor}
\institute{\talkaffiliation}
\date{\footnotesize{\emph{\href{\talkblog}{\talkemail}}}}
%-------------------------------------------------------------------------------------------------------
\begin{document}
\begin{frame}
\titlepage
\end{frame}
%-------------------------------------------------------------------------------------------------------
\section{¿Qué no es bullying?}
\begin{frame}[c]\frametitle{Se trata de tramos etáreos}
  \begin{alertblock}{Entre adultos, agresión}
    \pause No hablamos de bullying entre adultos, aquello es regulado por leyes civiles.
  \end{alertblock}
\pause
  \begin{alertblock}{Desde un adulto hacia un menor de edad}
    \pause Tampoco confundir con ``maltrato escolar'', que es la agresión del educador
    hacia el educando (alumno/a).
  \end{alertblock}
\end{frame}
%-------------------------------------------------------------------------------------------------------
\begin{frame}
  \frametitle{Malos sinónimos}
  \begin{alertblock}{}
    El bullying o acoso escolar no es lo mismo que mala conducta, conflicto escolar, conducta agresiva, conducta violenta o maltrato escolar. 
  \end{alertblock}
\end{frame}
%-------------------------------------------------------------------------------------------------------
\begin{frame}
  \frametitle{Conflictos escolares}
  \begin{exampleblock}{Son interpersonales}
    Los conflictos escolares se resuelven en la inmediatez, pues son concernientes a relaciones interpersonales. No requieren intervención de profesionales. No alteran la armonía escolar.
  \end{exampleblock}
\end{frame}
%-------------------------------------------------------------------------------------------------------
\begin{frame}
  \frametitle{No son ejemplos}
  \begin{alertblock}{Las siguientes acciones no constituyen ejemplos de bullying}
    \begin{itemize}
      \pause
    \item<+-> armar equipos de trabajo
    \item<+-> pelearse con el mejor amigo
    \item<+-> una pelea aislada en el patio
    \item<+-> no admitir a alguien en Facebook
    \item<+-> tener un grupo de ``mejores amigos''
    \item<+-> ponerse apodos entre todos
    \item<+-> ir al cine y no invitar a todos
    \item<+-> no invitar a alguien a mi casa
    \item<+-> no pasarle la tarea a alguien   
    \end{itemize}
  \end{alertblock}
\end{frame}
%-------------------------------------------------------------------------------------------------------
\section{¿Qué si es bullying?}
\begin{frame}
  \frametitle{Si son ejemplos}
  \begin{exampleblock}{Las siguientes acciones si constituyen ejemplos de
      bullying}
    \pause
    \begin{itemize}
    \item<+-> hacer comentarios despectivos sobre un compañero o su familia
    \item<+-> humillarlo delante de los demás
    \item<+-> organizar un plan el día de su cumpleaños para que los demás
        tengan que elegir
    \item<+-> organizar a un grupo de compañeros para que no sean amigos de alguien y no le hablen, no lo acompañen, no asistan a su cumpleaños, etc.   
    \end{itemize}
  \end{exampleblock}
\end{frame}
%-------------------------------------------------------------------------------------------------------
\section{Características del bullying}
\begin{frame}
  \frametitle{¿Cómo identificarlo?}
  \begin{block}{Algunas características del bullying son}
    \pause
    \begin{itemize}
    \item<+-> implica desbalance de poder
    \item<+-> tiende a ser frecuente
    \item<+-> puede ser físico, verbal o indirecto
    \end{itemize}
  \end{block}
\end{frame}
%-------------------------------------------------------------------------------------------------------
\begin{frame}
  \frametitle{Tipos de bullying}
  \begin{exampleblock}{El matonaje, bullying o agresión la podemos encontrar en diversos tipos, entre ellos:}
    \pause
    \begin{itemize}
    \item<+-> bloqueo social
    \item<+-> hostigamiento
    \item<+-> manipulación social
    \item<+-> coacción a conductas
    \item<+-> matonaje racial
    \end{itemize}
  \end{exampleblock}
\end{frame}
%-------------------------------------------------------------------------------------------------------
\begin{frame}
  \frametitle{Tipos de bullying}
  \begin{exampleblock}{El matonaje, bullying o agresión la podemos encontrar en diversos tipos, entre ellos:}
    \begin{itemize}
    \item<+-> matonaje sexual
    \item<+-> exclusión social
    \item<+-> amenaza a la integridad
    \item<+-> intimidación
    \item<+-> matonaje cibernético
    \item<+-> privar a otros de su propiedad, robarla o malograrla
    \end{itemize}
  \end{exampleblock}
\end{frame}
%-------------------------------------------------------------------------------------------------------
\section{Ley de convivencia escolar 20.536}
\begin{frame}
  \frametitle{Regula la violencia escolar en Chile}
  \begin{block}{Ley 20.536}
    En nuestro país, existe la LEY 201536 SOBRE VIOLENCIA ESCOLAR, promulgada el 17 de septiembre de 2011 y que especifica en su párrafo 3º, sobre convivencia escolar:
  \end{block}
\end{frame}
%-------------------------------------------------------------------------------------------------------
\begin{frame}
  \frametitle{Vigente desde 2011}
  \begin{block}{Artículo 16 B}
    Se entenderá por acoso escolar toda acción u omisión constitutiva de agresión u hostigamiento reiterado, realizada fuera o dentro del establecimiento educacional por estudiantes que, en forma individual o colectiva, atenten en contra de otro estudiante, valiéndose para ello de una situación de superioridad o de indefensión del estudiante afectado, que provoque en este último, maltrato, humillación o fundado temor de verse expuesto a un mal de carácter grave, ya sea por medios tecnológicos o cualquier otro medio, tomando en cuenta su edad y condición.
  \end{block}
\end{frame}
%-------------------------------------------------------------------------------------------------------
\section{Malas prácticas}
\begin{frame}
  \frametitle{Engendran estrategias de sobrevivencia}
  \begin{alertblock}{Malas prácticas, malos consejos, errores comunes, estrategias inútiles, expresiones inapropiadas:}
    \begin{itemize}
    \item<+-> aprende a defenderte (odio por odio = más violencia)
    \item<+-> dense la mano y háganse amigos -sin hablar- (aumenta el odio y
        refuerza humillación)
      \item<+-> es molestoso, le gusta que lo molesten, no es amistoso, le gusta llamar la atención, por todo reclama, nunca está contento, …​ califican como hostigamiento, el niño desarrolla entonces estrategias para sobrevivir/defenderse y la situación empeora   
    \end{itemize}
  \end{alertblock}
\end{frame}
%-------------------------------------------------------------------------------------------------------
\section{Citas}
\begin{frame}
  \frametitle{Citas, ejemplos, consejos}
\begin{block}{}
  Si un perro te muerde y tu muerdes al perro, ambos deben usar un bozal.
\end{block}
\end{frame}
%-------------------------------------------------------------------------------------------------------
\begin{frame}
  \frametitle{Citas, ejemplos, consejos}
  \begin{block}{}
    Usa las cosas y ama a las personas, lo inverso es inútil.
  \end{block}
\end{frame}
%-------------------------------------------------------------------------------------------------------
\begin{frame}
  \frametitle{Citas, ejemplos, consejos}
  \begin{block}{}
    No dejes que aplasten tus sueños: si tienes alma de dentista sé dentista, no oficinista: las aves asadas no vuelan.
  \end{block}
\end{frame}
%-------------------------------------------------------------------------------------------------------
\begin{frame}
  \frametitle{Citas, ejemplos, consejos}
  \begin{block}{}
    Aceptemos la diversidad y la protesta: los pájaros hambrientos y lo pájaros con el vientre lleno no pueden volar juntos.
  \end{block}
\end{frame}
%-------------------------------------------------------------------------------------------------------
\begin{frame}
  \frametitle{Citas, ejemplos, consejos}
  \begin{block}{}
    Denuncia el hostigamiento: Los pájaros nacidos en jaulas creen que volar es una enfermedad. 
  \end{block}
\end{frame}
%-------------------------------------------------------------------------------------------------------
\begin{frame}
  \frametitle{Citas, ejemplos, consejos}
  \begin{block}{}
    Los empates morales no existen: dos malas no hacen una buena.
  \end{block}
\end{frame}
%-------------------------------------------------------------------------------------------------------
\begin{frame}
  \frametitle{Citas, ejemplos, consejos}
  \begin{block}{}
    Diferencie entre problema y persona: con los problemas enfático, con las personas amable y respetuosa.
  \end{block}
\end{frame}
%-------------------------------------------------------------------------------------------------------
\begin{frame}
  \frametitle{Citas, ejemplos, consejos}
  \begin{block}{}
    Víctima y victimario siguen siendo personas: no existen humanos de segunda clase.
  \end{block}
\end{frame}
%-------------------------------------------------------------------------------------------------------
\begin{frame}
  \frametitle{Citas, ejemplos, consejos}
  \begin{block}{}
    No forzar nada, nunca.
  \end{block}
\end{frame}
%-------------------------------------------------------------------------------------------------------
\begin{frame}
  \frametitle{Citas, ejemplos, consejos}
  \begin{block}{}
    Enseña a tu hijo a reconocer apariencias: el agua en la botella cree que tiene forma de botella.
  \end{block}
\end{frame}
%-------------------------------------------------------------------------------------------------------
\begin{frame}
  \frametitle{Citas, ejemplos, consejos}
  \begin{block}{}
    Educa para ser feliz.
  \end{block}
\end{frame}
%-------------------------------------------------------------------------------------------------------
\begin{frame}
  \frametitle{Citas, ejemplos, consejos}
  \begin{block}{}
    Enséñale a ser agradecido.
  \end{block}
\end{frame}
%-------------------------------------------------------------------------------------------------------
\begin{frame}[c]\frametitle{Apéndice}
\centering\decofourleft\quad\decofourright

\textbf{\emph {Saludos a todas/os, ¡Muchas gracias!}}

Una copia del presente trabajo, se encuentra en el enlace \framedhref{\talkpubpdf}{\talkpubpdf}.

\end{frame}
\end{document}
%-------------------------------------------------------------------------------------------------------
% !TEX program = xelatex